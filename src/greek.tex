\chapter{Εκτεταμένη Ελληνική Περίληψη}

Στο παρόν κεφάλαιο ακολουθεί μία εκταταμένη ελληνική παρουσίαση του περιεχόμενου αυτής της διπλωματικής. Τα υποκεφάλαια έχουν την ίδια δομή με αυτή της αγγλικής εκδοχής και ο αναγνώστης παραπέμπεται στα αντίστοιχα σημεία της για ορισμένες αποδείξεις και λεπτομέρειες που έχουν παραληφθεί. 


\section{Ιδιωτικότητα στα Συστήματα Πληρωμών}
Η ιδότητα της ιδιωτικότητας είναι πολύ σημαντική για τα συστήματα πληρωμών, καθώς οι συναλλαγές περιέχουν πολλές φορές ευαίσθητα προσωπικά δεδομένα των χρηστών. Η διάρευσή του μπορεί να προκαλέσει διακρίσεις εις βάρος των χρηστών, καθώς και επιτρέπει της front-running επιθέσεις και τη δημιουργία "tainted" νομισμάτων.

Η πρώτη προσσέγγιση για την επιτεύξη ιδιωτικότητας, η οποία χρησιμοποιείται από το Bitcoin καθώς και τα περισσότερα κρυπτονομίσματα, είναι μέσω ψευδωνύμων. Κάθε χρήστης μπορεί να κατέχει πολλές διαφορετικές διευθύνσεις οι οποίες σε πρώτη όψη δε σχετίζονται με την πραγματική του ταυτότητα. 

Όμως, έχει αποδειχθεί από διάφορες μελέτες ότι οποισδήποτε μπορεί παρακολουθώντας τα δημόσια στοιχεία του blockchain σε συνδυασμό με τη συμπεριφορά των χρηστών να συνδέσει τις διάφορες διευθύνσεις κάθε χρήστη μεταξύ τους και να ξεχωρίσει σε ποιον χρήστη ανήκουν.

Για αυτόν τον λόγο, προτάθηκαν συστήματα πληρωμών που παρέχουν ισχυρότερες εγγυήσεις ιδιωτικότητας. Σε αυτά τα συστήματα η έννοια της ιδιοτηκότητας περιέχει δύο ιδιότητες:
\begin{itemize}
    \item \textbf{Ανωνυμία}: Απόκρυψη των ψευδώνυμων των πραγματικών συμμετεχώντων σε μια συναλλαγή. Επιτυγχάνεται μέσω των συνόλων ανωνυμίας (anonymity sets) τα οποία εισάγουν και άλλα ψευδόνυμα για να αποκρύψουν στη συναλλαγή για να αποκρύψουν εκείνα του αποστολέα και των παραληπτών.
    \item \textbf{Εμπιστευτικότητα}: Απόκρυψη του ποσού που μεταφέρεται σε κάθε συναλλαγή και κατά επέκταση των υπολοίπων των χρηστών. Επιτυγχάνεται μέσω της χρήσης ομομορφικών δεσμεύσεων (commitments).
\end{itemize}

Η εγκυρότητα των συναλλαγών στα συστήματα αυτά ελέγχεται από αποδείξεις μηδενικής γνώσης. Μέσω αυτλων των αποδείξεων, δεν γνωστοποιείται κάποια επιπλέον πληροφορία αλλά ταυτόχρονα οι υπόλοιποι χρήστες μπορούν να επιβεβαιώσουν ότι ισχύουν κάποιες προϋπόθέσεις απαραίτητες για την εγκυρότητα των συναλλαγών.

Παρακάτω παρατίθεται μία σύντομη περιγραφή ιδιωτικών συστημάτων πληρωμών. Περισσότερες λεπτομέρειες για κάθε σύστημα μπορούν να βρεθούν στο αντίστοιχο σημείο της αγγλικής εκδοχής.

\subsubsection{Zerocash}
Το Zerocash επιτυγχάνει την ιδιωτικότητα μέσω του κρυπτογραφικού εργαλείου των σύντομων αποδείξεων μηδενικής γνώσης zk-SNARKs. Όλα τα νομίσματα που δημιουργούνται στο Zerocash αποθηκεύονται σε μια δομή δεδομένων (merkle tree) και περιέχουν με κρυπτογραφημένο τρόπο ένα μοναδικό σειριακό αριθμό, την τιμή τους και την διεύθυνση που αντιστοιχεί στον κατοχό τους. Οι χρήστες αντί να εισάγουν σαν είσοδο στη συναλλαγή το ίδιο το νομισμά αποδεικνύουν ότι κατέχουν ένα από τα νομίσματα που είναι αποθηκευμένο στη δομή χωρίς να το προσδιορίσουν συγκεκριμένα. Για να το ξοδέψουν δημοσιουποιούν μαζί με την απόδειξη και το μοναδικό σειριακό αριθμό ώστε να μην υπάρχει η δυνατότητα για double spending. Τέλος δημιουργούν ένα καινούργιο νομισμά για τον παραλήπτη με την ίδια τιμή.

Το Zerocash παρέχει την μέγιστη δυνατή ανωνυμία καθώς το anonymity set περιλαμβάνει όλα τα πιθανά νομίσματα. Όμως, έχει δύο σημαντικά αρνητικά στοιχεία. Το πρώτο είναι ότι τα zk-SNARKs επιβάλλουν trusted setup. Σε περίπτωση που για κάποιον λόγο διαρεύση σε κάποιον οι πληροφορίες του setup τότε αυτός μπορεί να δημιουργήσει ορθές αποδείξεις χωρίς να κατέχει την απαραίτητη μυστική πληροφορία. Το δεύτερο αρνητικό είναι ότι καθώς δε μπορεί να διακριθεί ποιο νόμισμα ξοδεύτηκε σε κάθε συναλλαγή, η δομή δεδομένων μεγαλώνει σε κάθε συναλλαγή. Αυτό έχει σαν αποτέλεσμα κάθε κόμβος να πρέπει να αποθηκεύει ένα σημαντικά μεγάλο όγκο δεδομένων.

\subsubsection{Monero}
Το Monero αποτελεί και αυτό ένα ιδιωτικό σύστημα πληρωμών. Σε αντίθεση με το Zerocash δε χρησιμοποιεί zk-SNARKs αλλά επιτυγχάνει την ανωνυμία μέσω της χρήσης των υπογραφών δακτυλίου (ring singatures) και των κρυφών διευθύνσεων (stealth address) και την εμπιστευτικότητα μέσω των ring confidential transaction. 

Οι υπογραφές δακτυλίου επιτρέπουν σε ένα μέλος μιας ομάδας να υπογράψει εκ μέρους όλης της ομάδας. Οι υπόλοιποι χρήστες μπορούν να επιβεβαιώσουν ότι η υπογραφή ανήκει σε κάποιο μέλος της ομάδας χωρίς όμως να μπορούν να ξεχωρίσουν το συγκεκριμένο μέλος. Αυτό το εργαλείο προστατεύει την ανωνυμία του αποστολέα.

Για την ανωνυμία του παραλήπτη χρησιμοποιούνται οι κρυφές διευθύνσεις (stealth addresses). Αυτές αποτελούν ένα κρυπτογραφικό εργαλείο που  επιτρέπει τη δημιουργία μιας καινούργιας μίας χρήσης διεύθυνσης για τον αποστολέα, η οποία προκύπτει από το δημόσιο κλειδί του αλλά ένας εξωτερικός χρήστης δε μπορεί να την συνδέσει με αυτό.

Το Monero, παρόμοια με το Zerocash, έχει το ίδιο πρόβλημα του απαιτούμενου χώρου που πρέπει να διατηρεί κάθε κόμβος του συστήματος.

\subsubsection{Quisquis}
Το Quisquis αποτελεί μια υλοποίηση ιδιωτικού συστήματος πληρωμών, η οποία επιλύει το παραπάνω πρόβλημα του απαιτούμενου χώρου. Το πετυχαίνει με τη χρήση των Updatable Public Keys. Μέσω αυτού του κρυπτογραφικού εργαλείου οι χρήστες μπορούν να κατασκευάζουν πολλαπλά δημόσια κλειδιά που μοιράζονται όμως ένα ιδιωτικό κλειδί. Τα δημόσια κλειδιά παρόλου που κατασκευάζονται από το ίδιο ιδωτικό κλειδί δεν μπορούν να συσχετιστούν από έναν εξωτερικό παρατηρητή (ο οποίος δε γνωρίζει το ιδιωτικό κλειδί). 

\textbf{Updatable Public Keys}:
Πιο συγκεκριμένα ένα σχήμα ανανεώσιμων δημόσιων κλειδιών (UPK scheme) περιέχει τις εξής ιδιότητες:
\begin{itemize}
    \item \textbf{Ορθότητα}: Όλα τα κλειδιά, που έχουν κατασκευαστεί από τίμιους χρήστες, επαληθεύονται σωστά.
    \item \textbf{Αντίσταση στη διακρισιμότητα}: Ένας αντίπαλος δεν μπορεί να διακρίνει μεταξύ ενός καινούργιου δημόσιου κλειδιού και μιας ενημερωμένης έκδοσης ενός δημόσιου κλειδιού που ήδη γνωρίζει
    \item \textbf{Άντισταση στην πλαστογράφιση}: Ένας αντίπαλος δεν μπορεί να μάθει το μυστικό κλειδί ενός ενημερωμένου δημόσιου κλειδιού χωρίς να γνωρίζει το μυστικό κλειδί του αρχικού δημόσιου κλειδιού.
\end{itemize}

Στο Quisquis οι συναλλαγές περιέχουν ως είσοδο λογαριασμούς, που αποτελούνται από μία διεύθυνση (ένα δημόσιο κλειδί-upk) και το υπόλοιπο πο σχετίζεται με τον λογαριασμό αυτών, του αποστολέα, των παραληπτών καθώς και άλλων χρηστών που λειτουργόυν ως anonymity set. Με το που χρησιμοποιηθεί ένας λογαριασμός σε μία συναλλαγή καταναλώνεται και δημιουργείται στη θέση του ένας καινούργιος με νέα διεύθυνση που προκύπτει από την ανανέωση του εκάστοτε προηγούμενου δημοσίου κλειδιού. Έτσι οι χρήστες χρειάζεται να αποθηκεύουν μόνο την τελευταία έκδοση των λογαριασμών και όχι όλη την πληροφορία που περιέχει και τους λογαριασμούς που έχουν ήδη "ξοδευτεί".

\section{Έλεγχος στα Ανώνυμα Συστήματα Πληρωμών}
Τα πλήρη ιδιωτικά συστήματα πληρωμών, ενώ έλυσαν το πρόβλημα της προστασίας των ιδιωτικών δεδομένων, δημιούργησαν κάποια νέα προβλήματα καθώς δεν είναι πλέον εφικτός ο έλεγχος των χρηστών στα συγκεκριμένα συστήματα. Αυτό είχε ως αποτέλεσμα να χρησιμοποιούνται από κακόβουλους χρήστες για την διεξαγωγή παράνομων δραστηριοτήτων, όπως για παράδειγμα "ξέπλυμα" χρημάτων ή παράνομο εμπόριο. Έτσι, έχει απαγορευτεί από διάφορους οργανισμούς η χρήση πολλών από αυτών των συστημάτων.

Επομένως, για την κάλυψη αυτού του κενού δημιουργήθηκαν συστήματα που προσπαθούν να συνδυάσουν την ιδιωτικότητα με την δυνατότητα ελέγχου. Στη βιβλιογραφία διακρίνονται δύο βασικοί τρόποι για την επίτευξη του παραπάνω στόχου, η εισαγωγή μιας κεντρικής αρχής (εμπιστή τρίτη οντότητα) ή η εισαγωγή γενικών ελεγκτών (general auditor).

\subsubsection{Κεντρική Έμπιστη Αρχή}
Ο πιο απλός τρόπος για την εφαρμογή της δυνατότητας ελέγχου σε ιδιωτικά συστήματα είναι η εισαγωγή μιας κεντρικής αρχής ή μιας ομάδας κεντρικών αρχών (multi-party computation). Σύμφωνα με αυτή τη προσέγγιση, οι χρήστες ενσωματώνουν επιπλέον πληροφορίες στις συναλλαγές, οι οποίες κρυπτογραφούνται με το δημόσιο κλειδί ενός καθορισμένου αξιόπιστου ελεγκτή. Έτσι, τα δεδομένα των χρηστών παραμένουν κρυφά για τους υπόλοιπους συμμετέχοντες στο σύστημα, εκτός από την κεντρική αρχή, η οποία μπορεί να αποκρυπτογραφήσει τις βοηθητικές πληροφορίες ανά πάσα στιγμή χωρίς τη συγκατάθεση των χρηστών.

Ένα παράδειγμα τέτοιου συστήματος είναι το Zcash extension. Το σύστημα αυτό επεκτείνει το Zerocash προσθέντωντας επιπλέον πληροφορίες σε κάθε νόμισμα. Σε αυτές περιλαμβάνονται οι counters που αποθηκεύουν αθροιστικές πληροφορίες για κάθε χρήστη και χρησιμοποιούνται για την επιβολή φορολογίας και ορίων συναλλαγών, καθώς και απαραίτητες πληροφορίες για τον ιχνηλάτηση των συναλλαγών. Οι επιπλέον πληροφορίες είναι κρυπτογραφημένες με το κλειδί μιας έμπιστης τρίτης οντότητας, η οποία μπορεί οποιαδήποτε στιγμή να δει το περιεχόμενό τους. Βέβαια, το Zcash extension δίνει την δυνατότητα στους χρήστες να γνωρίζουν πότε έχει αρθεί η ανωνυμία τους από την αρχή.

Το κύριο αρνητικό αυτής της προσέγγισης είναι ότι όλη η πληροφορία συγκεντρώνεται σε μία κεντρική αρχή, με αποτέλεσμα η τελευταία να αποκτά μεγάλη δύναμη.  Το γεγονός αυτό μπορεί να έχει αρνητικό αντίκτυπο στη προστασία της ιδιοτηκότητας των χρηστών.

\subsubsection{General Auditor}
Για να αποφευχθεί η συλλογή όλων των πληροφοριών σε μια κεντρική αρχή (ή ομάδα αρχών), προτάθηκε μια δεύτερη προσέγγιση. Πρόκειται για ένα διαδραστικό πρωτόκολλο μεταξύ του ελεγχόμενου χρήστη και του ελεγκτή. Σε αυτή την περίπτωση, ο ελεγκτής, ο οποίος μπορεί να είναι οποιαδήποτε αρχή ελέγχου, μπορεί να θέσει συγκεκριμένες ερωτήσεις που προέρχονται από τις πολιτικές του συστήματος. Οι χρήστες απαντούν σε αυτές τις ερωτήσεις με αποδείξεις μηδενικής γνώσης που βασίζονται σε δεδομένα αποθηκευμένα on-chain. Το πρωτόκολλο αυτό προϋποθέτει τη συγκατάθεση και τη συνεργασία του ελεγχόμενου χρήστη. Ωστόσο, η απαίτηση αυτή δεν μπορεί να γίνει αντικείμενο εκμετάλλευσης από μη συμμορφούμενους χρήστες, δεδομένου ότι η άρνηση συνεργασίας με τις αρχές μπορεί να θεωρηθεί ισοδύναμη με αποτυχημένο έλεγχο.

Παρακάτω παρατίθεται μία σύντομη περιγραφή τέτοιων συστημάτων. Περισσότερες λεπτομέρειες για κάθε σύστημα μπορούν να βρεθούν στο αντίστοιχο σημείο της αγγλικής εκδοχής.

\subsubsection{zkLedger}
Το zkLedger επιτυγχάνει να παρέχει πλήρη ιδιωτικότητα στους χρήστες (ανωνυμία και εμπιστευτικότητα) και παράλληλα την δυνατότητα ελέγχου κατά τον οποίο η αρχή μαθαίνει μόνο την απάντηση σε συγκεκριμένες ερωτήσεις και όχι την συνολική πληροφοριά των χρηστών. Αυτό το πετυχαίνει μέσω της ειδικής μορφής ledger που προτείνει. Αυτή η μορφή είναι ένας πίνακας όπου οι συναλλαγές αντιστοιχούν σε γραμμές και οι χρήστες σε στήλες. Κάθε συναλλαγή περιλαμβάνει πληροφορίες για όλους τους άλλους χρήστες, ακόμη και για εκείνους που δεν συμμετέχουν. Για την απόκρυψη των μεταβιβαζόμενων ποσών καθώς και του υπολοίπου που κατέχει κάθε χρήστης κάθε εγγραφή σε μια συναλλαγή περιέχει μια δέσμευση σε μια αξία που χρεώνεται ή πιστώνεται στον χρήστη. Όλες οι καταχωρήσεις για τους μη συμμετέχοντες έχουν δεσμευμένη τιμή 0. Λόγω της ιδιότητας απόκρυψης των δεσμεύσεων ένας αντίπαλος δεν μπορεί να διακρίνει μεταξύ μιας μηδενικής και μιας μη μηδενικής δεσμευμένης τιμής. 

Η διαδικασία ελέγχου παρουσιάζεται μέσα από το επόμενο παράδειγμα. Ένας ελεγκτής μπορεί να ρωτήσει έναν χρήστη «Πόσα ευρώ έχετε στην κατοχή σας τη στιγμή t;». Ο χρήστης απαντά με μία τιμή περιέχωντας μαζί και μια απόδειξη μηδενικής γνώσης που εξασφαλίζει την εγκυρότητά της απάντησής του. Ο ελεγκτής μπορεί να πολλαπλασιάσει όλες τις δεσμεύσεις στο λογιστικό βιβλίο για τον ελεγχόμενο χρήστη και μπορεί να επαληθεύσει αν η απόδειξη και η απάντηση είναι έγκυρες. Δεδομένου ότι η στήλη του βιβλίου αντιπροσωπεύει όλα τα ποσά που έχει λάβει ή δαπανήσει ο αντίστοιχος χρήστης, ο ελεγκτής μπορεί να είναι βέβαιος ότι ο χρήστης δεν θα μπορούσε να αποκρύψει καμία από τις συναλλαγές του κατά τη διάρκεια του ελέγχου.

Το κύριο αρνητικό αυτού του συστήματος είναι ότι κάθε συναλλαγή περιέχει όλους τους χρήστες και είναι απαραίτητο να είναι γνωστή η προηγούμενη κατάσταση όλων των χρηστών για να μπορεί να πραγματοποιηθεί η συναλλαγή. Επομένως, πολλαπλοί χρήστες δεν μπορούν να παράγουν παράλληλα διαφορετικές συναλλαγές, καθώς οι ταυτόχρονες συναλλαγές έχουν πάντα συγκρούσεις.

\subsubsection{PGC}
Το PGC ακολουθεί μία διαφορετική προσέγγιση. Για να επιτύχει αποδοτικό έλεγχο των χρηστών παραχωρεί την ανωνυμία. Δηλαδή παρέχει μόνο την ιδιότητα της εμπιστευτικότητας (confidentiality) όσον αφορά την προστασία της ιδιοτηκότητας των χρηστών. Η ανωνυμία στηρίζεται πάλι στην χρήση απλών ψευδωνύμων. Με αυτών τον τρόπο η αρχή μπορεί να μάθει ποιοι χρήστες συμμετέχουν σε κάθε συναλλαγή αλλά δεν μαθαίνει το ποσό που συναλλάσεται καθώς και το υπόλοιπο των χρηστών. 

Το PGC παρέχει τρία είδη πολιτικών (ερωτήσεων) για τον έλεγχο τον χρηστών. 
\begin{itemize}
    \item \textbf{Πολιτική ορίου (limit policy)}: περιορίζει το όριο των χρημάτων που μπορεί να μεταφερθεί από και προς ένα χρήστη για ένα χρονικό διάστημα.
    \item \textbf{Φορολογία (tax policy)}: Ένα ποσό των χρημάτων που δέχεται ένας χρήστης σε κάθε συναλλαγή ή σε ένα χρονικό διάστημα πρέπει να αποστέλεται στο tax office.
    \item \textbf{Πολιτική Επιλεκτικής Αποκάλυψης (open policy)}: Αποκάλυψη της τιμής που μεταφέρθηκε σε κάποια συναλλαγή.
\end{itemize}

Ο έλεγχος πραγματοποιείται ως εξής: Η αρχή διαλέγει μία διεύθυνση και συλλέγει όλες τις συναλλαγές στις οποίες συμμετείχε η διεύθυνση αυτή για το επιλεγμένο χρονικό διάστημα. Χάρη στην ιδιότητα του προσθετικού ομομορφισμού των δεσμεύσεων που χρησιμοποιούνται στο PGC, η αρχή μπορεί να βρεί το άθροισμα όλων των ζητούμενων ποσών για να εφαρμόσει τις πολιτικές limit και tax. Παράλληλα μπορεί να ζητήσει από τον χρήστη να αποκαλύψει την τιμή από κάποια ζητούμενη συναλλαγή. Ο χρήστης απαντά στις ερωτήσεις της αρχής με αποδείξεις μηδενικής γνώσης, με αποτέλεσμα να μην διαρρέεται κάποια επιπλέον πληροφορία.

Προφανώς το κυριότερο αρνητικό αυτού του συστήματος είναι η έλλειψη της ανωνυμίας των χρηστών.


\section{AQQUA: Επέκταση του Quisquis με την Δυνατότητα Ελέγχου}

Στην εργασία αυτή προτείνουμε μία νέα προσέγγιση για την επίτευξη του συνδυασμού της ιδιωτικότητάς και του ελέγχου που έχει ως σκοπό την επίλυση των παραπάνω προβλημάτων.

Δηλαδή, στοχεύουμε στην κατασκευή ενός αποτελεσματικού, ανώνυμου, εμπιστευτικού και ελεγχόμενου συστήματος που μπορεί επίσης να υποστηρίζει ταυτόχρονες συναλλαγές και να διατηρεί ένα σταθερό μέγεθος του state ανεξάρτητα από τον αριθμό των χρηστών ή το ιστορικό των συναλλαγών. Για να δημιουργήσουμε ένα τέτοιο σύστημα, προτείνουμε το AQQUA, το οποίο επεκτείνει το σύστημα Quisquis με έναν γενικό ελεγκτή. Έτσι, το AQQUA συνδυάζει την ανωνυμία του Quisquis με την εκφραστικότητα της πολιτικής και τη ρύθμιση του PGC. Αυτό σημαίνει ότι ο ελεγκτής μπορεί να εκτελεί ερωτήματα σχετικά με το ανώτατο όριο του ποσού που αποστέλλεται/λαμβάνεται από τον χρήστη σε μια δεδομένη περίοδο, σχετικά με τη μη συμμετοχή ενός χρήστη σε μια δεδομένη συναλλαγή ή περίοδο, καθώς και σχετικά με την ακριβή αξία που αποστέλλεται/λαμβάνεται σε μια συναλλαγή.

Ο σχεδιασμός του AQQUA έπρεπε να ξεπεράσει την ακόλουθη βασική πρόκληση, ώστε να επιτρέψει τις λειτουργίες ελέγχου, διατηρώντας παράλληλα το απόρρητο του χρήστη. Λόγω της ιδιότητας της ανωνυμίας, οι χρήστες μπορούν να αποκρύψουν τους λογαριασμούς τους και κατά συνέπεια τα ποσά που είναι απαραίτητα για τη διαδικασία ελέγχου. Επιπλέον, δεδομένου ότι το Quisquis είναι permissionless και ιδιωτικό, μια αρχή δεν μπορεί να επιβάλει κάποιες αποτελεσματικές κυρώσεις για τους μη συμμορφούμενους χρήστες.

Για να ξεπεραστεί αυτή η πρόκληση, εισάγουμε μια λειτουργία εγγραφής στο Quisquis μέσω μιας αρχής εγγραφής. Αυτό σημαίνει ότι οι χρήστες πρέπει πρώτα να εγγραφούν στο συστήμα, παραδίδοντας την πραγματική τους ταυτότητα. Στη συνέχεια μπορούν να δημιουργήσουν νέους, μη συνδεδεμένους λογαριασμούς που χρησιμοποιούνται για συναλλαγές εντός του συστήματος. Η λειτουργία εγγραφής παρέχει στο σύστημα έναν τρόπο να γνωρίζει ποιοι χρήστες χρησιμοποιούν το σύστημα και να τους τιμωρεί με τρόπο που δεν εμπίπτει στο πεδίο εφαρμογής του συστήματος. Επιπλέον, το AQQUA χωρίζει το state σε δύο σύνολα. Διατηρεί το σύνολο UTXO που χρησιμοποιείται στο Quisquis, το οποίο περιέχει τους "αξόδευτους" λογαριασμούς του χρήστη, αλλά προσθέτει επίσης ένα νέο σύνολο που περιέχει τις δημόσιες πληροφορίες εγγραφής του χρήστη μαζί με τις απαραίτητες πληροφορίες για να διασφαλίσει ότι οι χρήστες δεν μπορούν να αποκρύψουν πληροφορίες κατά τη διαδικασία ελέγχου.

Πιο συγκεκριμένα, στην εργασία ορίζουμε αυστηρά τα στοιχεία ενός ελεγχόμενου και ιδιωτικού DPS όπως το AQQUA.

\subsubsection{Οντότητες - Entities}
Το AQQUA αποτελείται από τις εξής οντότητες:
\begin{itemize}
    \item Αρχή εγγραφής (RA): Ο ρόλος της  είναι να εγγράφει νέους χρήστες στο σύστημα. Οι χρήστες εγγράφονται στέλνοντας τις πραγματικές ταυτότητάς τους μαζί με ένα αρχικό δημόσιο κλειδί που δημιουργούν μόνοι τους. Η RA αποθηκεύει αυτές τις πληροφορίες off-chain. Όλοι οι λογαριασμοί με τους οποίους ο χρήστης πραγματοποιεί συναλλαγές θα προέρχονται από αυτό το αρχικό δημόσιο κλειδί, μέσω του μηχανισμού ανανέωσης του UPK σχήματος. Ο σκοπός της διαδικασίας εγγραφής είναι σημαντικός, καθώς δημιουργεί μια σύνδεση μεταξύ του δημόσιου κλειδιού ενός χρήστη και της πραγματικής του ταυτότητας, η οποία θα χρησιμοποιηθεί για την πιθανή τιμωρία των μη συμμορφούμενων χρηστών.
    \item Αρχή Ελέγχου (ΑΑ): Ο ρόλος της είναι να πραγματοποιεί τη διαδικασία ελέγχου προκειμένου να επαληθεύει ότι οι χρήστες συμμορφώνονται με τις πολιτικές του συστήματος. Εάν διαπιστωθεί ότι ένας χρήστης του συστήματος δεν συμμορφώνεται, η ΑΑ συνεργάζεται με την RA για την επιβολή των σχετικών κυρώσεων.
    \item Χρήστες (U): Χρήστες που πραγματοποιούν συναλλαγές μεταξύ τους.
\end{itemize}


\subsubsection{State}
Στο AQQUA, το state αποτελείται από τα ακόλουθα σύνολα:
\begin{itemize}
    \item \textbf{UTXOSet}: Ένας πίνακας που περιέχει τους "αχρησιμοποίητους" λογαριασμούς, δηλαδή τους λογαριασμούς που έχουν καταγραφεί ως έξοδοι μιας έγκυρης συναλλαγής, αλλά δεν έχουν (ακόμη) χρησιμοποιηθεί ως είσοδοι.
    \item \textbf{UserSet}: Ένας πίνακας που περιέχει για κάθε φυσικό χρήστη το αρχικό δημόσιο κλειδί του και μια δέσμευση για τον αριθμό των λογαριασμών που του ανήκουν.
\end{itemize}

\subsubsection{Λογαριασμοί - Accounts}
Οι λογαριασμοί χρηστών έχουν τη μορφή $\acct = (\pk, \comm{\varbl} , \comm{\varout} , \comm{\varin} )$, όπου $\varbl$ είναι το υπόλοιπο του λογαριασμού και $\varout, \varin$ είναι το συνολικό ποσό που έχει στείλει και λάβει ο λογαριασμός, αντίστοιχα. Οι παραπάνω τιμές αποθηκεύονται μέσα σε δεσμεύσεις ώστε να μην μπορεί κάποια εξωτερική οντότητα να μάθει την πραγματική τιμή τους.
Κάθε χρήστης μπορεί να έχει πολλούς λογαριασμούς οι οποίοι αποθηκεύονται στο UTXOSet. 

\subsubsection{Userinfo}
Κάθε χρήστης συνδέεται με μια πλειάδα της μορφής $\userinfo = (\inpk, \comm{\numaccs})$, η οποία αποθηκεύεται στο UserSet. Το δημόσιο κλειδί $\inpk$ είναι ένα αρχικό δημόσιο κλειδί που παρέχεται κατά την εγγραφή. Το δημόσιο κλειδί κάθε λογαριασμού που ανήκει στον χρήστη θα μοιράζεται το ίδιο μυστικό κλειδί με το $\inpk$.

Η τιμή $\numaccs$ είναι ο αριθμός των λογαριασμών στο UTXOSet που ανήκουν στον χρήστη και αποθηκεύεται ως δέσμευση, ώστε να παραμένει κρυφή. Η καταγραφή του αριθμού των λογαριασμών που κατέχει ένας χρήστης είναι απαραίτητη για την υποστήριξη πολιτικών που σχετίζονται με όρια τιμών, όπως το συνολικό ποσό που έχει λάβει ή στείλει ένας χρήστης σε μια χρονική περίοδο. Διαφορετικά, τέτοιες πολιτικές θα μπορούσαν εύκολα να παρακαμφθούν μέσω της δημιουργίας εικονικών ταυτοτήτων (sybil identities). Το άνοιγμα της δέσμευσης $\numaccs$ θα αποκαλυφθεί μόνο στον ΑΑ κατά τη διαδικασία ελέγχου.

\subsubsection{Πολιτικές Ελέγχου - Policies}
Ένα ελεγχόμενο DPS θα πρέπει να υποστηρίζει ένα πλούσιο σύνολο πολιτικών συμμόρφωσης. Αυτές μπορούν να αποτυπωθούν ως κατηγορήματα επί ενός αρχικού δημόσιου κλειδιού $\inpk$, μιας χρονικής περιόδου που αντιπροσωπεύεται από ένα state έναρξης $\snap_1$ και ένα state λήξης $\snap_2$, και βοηθητικών πληροφοριών $\aux$ που εξαρτώνται από την εκάστοτε πολιτική. Σε όλα τα κατηγορήματα, χρησιμοποιούμε τον συμβολισμό $A1, A2$ για να δηλώσουμε το σύνολο των λογαριασμών στην κατάσταση $\snap_1.\utxoset, \snap_2.\utxoset$ που ανήκουν στον ιδιοκτήτη του $\inpk$.

Οι πολιτικές που επιτρέπει το AQQUA είναι όι εξής:

\begin{itemize}
    \item Πολιτική ορίου αποστολής $f_{\sendlimit}$: Περιορίζει το συνολικό ποσό που μπορεί να στείλει ένας χρήστης του πραγματικού κόσμου μέσα σε μια συγκεκριμένη περίοδο. Μπορεί να καθορίζεται από την $\AA$ εκτός αλυσίδας και να ανακοινώνεται στον χρήστη για μια συγκεκριμένη περίοδο, ανάλογα με την εφαρμογή. Τα $\snap_1, \snap_2$ είναι οι καταστάσεις του blockchain στην αρχή και στο τέλος της περιόδου, αντίστοιχα.
    \begin{align*}
        f_{\sendlimit}(\inpk, (\snap_1, \snap_2), \srlimit) = 1 \iff 
        \left\{ \begin{aligned}
                            &  \left(\sum_{\acct\in A_2}{\varout} - \sum_{\acct \in A_1}{\varout} \right)
                            \leq \srlimit 
                 \end{aligned} \right\}
    \end{align*}
    όπου $\varout$ είναι το άνοιγμα της δέσμευσης $\comm{\varout}$ ενός λογαριασμού $\acct$, χρησιμοποιώντας το ιδιωτικό κλειδί $\sk$ του λογαριασμού.

    \item Πολιτική ορίου λήψης $f_{\receivelimit}$: Ομοίως, το συνολικό ποσό που μπορεί να λάβει ένας `φυσικός' χρήστης.
        \begin{align*}
           f_{\receivelimit}(\inpk, (\snap_1, \snap_2), \srlimit) = 1  \iff 
           \left\{ \begin{aligned} 
                    & \left(\sum_{\acct\in A_2}{\varin} - \sum_{\acct \in A_1}{\varin}\right) \leq \srlimit
            \end{aligned} \right\}
        \end{align*}
        όπου $\varin$ είναι το άνοιγμα του $\comm{\varin}$ για τον λογαριασμό $\acct$, που υπολογίζεται χρησιμοποιώντας το ιδιωτικό κλειδί του λογαριασμού $\sk$.


    \item Open policy $f_{\open}$: Αποκάλυψη της αξίας του ποσού που αποστέλλεται ή λαμβάνεται από έναν χρήστη σε μια συναλλαγή.
    \begin{align*}
        f_{\open}(\inpk, (\snap_1, \snap_2), v_{\open}) = 1 \iff 
        \left\{ \begin{aligned}   
        & (v = \left(\sum_{\acct \in A_{2}}\varbl - \sum_{\acct \in A_{1}}\varbl \right)\in \vset \\
        &\lor v = \left(\sum_{\acct \in A_{1}}\varbl - \sum_{\acct \in A_{2}}\varbl \right)\in \vset )\\
        & \land v = v_{\open} 
    \end{aligned} \right\}
    \end{align*}
    όπου $\varbl$ είναι το άνοιγμα του $\comm{\varbl}$ ενός $\acct$.

    \item Όριο αξίας συναλλαγής $f_{\txnlimit}$: Ανώτατο όριο του συνολικού μεταφερόμενου ποσού που μπορεί να σταλεί σε μια συναλλαγή.
    \begin{align*}
        f_{\txnlimit}(\inpk, (\snap_1, \snap_2), v_{\max}) = 1 \iff  
        \left\{ \begin{aligned} 
        & v = \left(\sum_{\acct \in A_1}\varbl - \sum_{\acct \in A_2}\varbl \right) \leq v_{\max} 
        \end{aligned} \right\}
    \end{align*}

    \item Μη συμμετοχή $f_{\nonpart}$: Μη συμμετοχή σε μια συγκεκριμένη συναλλαγή $\txn$ ή αδράνεια του χρήστη για ένα χρονικό διάστημα. Οι καταστάσεις $\snap_1, \snap_2$ είναι οι καταστάσεις πριν και μετά την εφαρμογή μιας συναλλαγής ή στην αρχή και στο τέλος της περιόδου.
        \begin{align*}
            f_{\nonpart}(\inpk, (\snap_1, \snap_2)) = 1 \iff  
            \left\{ 
                \begin{aligned} 
                    % \land & \left(\sum_{\acct \in A_{1}}\varbl - \sum_{\acct \in A_{2}}\varbl \right) = 0 \\
                    \land & \left(\sum_{\acct \in A_{1}}\varout - \sum_{\acct \in A_{2}}\varout \right) = 0 \\
                    \land &\left(\sum_{\acct \in A_{1}}\varin - \sum_{\acct \in A_{2}}\varin \right) = 0
                \end{aligned} 
            \right\}
        \end{align*}

\end{itemize}

\subsubsection{Συναρτήσεις - Functionalities}
Ένα ελεγχόμενο ιδιωτικό αποκεντρωμένο σύστημα πληρωμών είναι μια πλειάδα αλγορίθμων πολυωνυμικού χρόνου που ορίζονται ως εξής:

\begin{itemize}
    \item $(\instate, \params) \gets \setup(\secpar)$: 
    Δημιουργεί την αρχική κατάσταση του συστήματος $\instate$ και τις δημόσιες παραμέτρους $\params$, οι οποίες δίνονται έμμεσα ως είσοδος σε όλους τους άλλους αλγορίθμους.

    \item $(\sk, \userinfo, \acct, \pi) \gets \reg()$: 
    Χρησιμοποιείται από έναν χρήστη για να δημιουργήσει τις πληροφορίες εγγραφής $\userinfo$ και τον πρώτο του λογαριασμό $\acct$.

    \item $0/1 \gets \verreg(\userinfo, \acct, \pi, \state)$: 
        Χρησιμοποιείται από την Αρχή Εγγραφής για την επαλήθευση των πληροφοριών εγγραφής και του λογαριασμού ενός χρήστη.

    \item $\state' \gets \applyreg(\userinfo, \acct, \state)$:
    Χρησιμοποιείται από την Αρχή Εγγραφής για την προσθήκη ενός χρήστη στο σύστημα μετά την επιτυχή εγγραφή του.
    
    \item $\txn = (\{\acct\}_{i=1}^n, \{{\acct}'\}_{i=1}^n, \pi) \gets \trans(\sk,\sset, \rset, \vv{\v_\sset}, \vv{\v_\rset}, \anset)$:
    Χρησιμοποιείται από τον αποστολέα (κάτοχο του ιδιωτικού κλειδιού $\sk$) για να δημιουργήσει μια συναλλαγή που ανακατανέμει τα νομίσματά του από τους λογαριασμούς του στο $\sset$ στους λογαριασμούς των παραληπτών στο $\rset$. Τα διανύσματα $\vv{\v_\sset},\vv{\v_\rset}$ περιγράφουν τις αλλαγές στις τιμές των $\sset, \rset$ αντίστοιχα. Για την απόκρυψη των συμμετεχόντων λογαριασμών, ένα σύνολο ανωνυμίας $\anset$ (anonymity set) δίνεται ως είσοδος.

    \item $\txnca = (\acct, \{\userinfo_i\}_{i=1}^n, \{\userinfo_i^\prime\}_{i=1}^n, \pi)\gets \createAcct(\userinfo, \anset)$: \\
    Δημιουργεί μια συναλλαγή για τη δημιουργία ενός νέου λογαριασμού για τον ιδιοκτήτη του $\userinfo.\inpk$ και ενημερώνει κατάλληλα την τιμή της δέσμευσης για τον αριθμό των λογαριασμών που κατέχει, $\userinfo.\com_{\numaccs}$.
    Για την απόκρυψη της σύνδεσης μεταξύ του νέου λογαριασμού $\acct$ και του αντίστοιχου $\inpk$, δίνεται ένα σύνολο ανωνυμίας $\anset$.

    \item $\txnda = (\{\acct\}_{i=1}^n, \{{\acct}'\}_{i=1}^n, \{\userinfo\}_{i=1}^n, \{{\userinfo}'\}_{i=1}^n  \pi)$\\ $\gets \deleteAcct(\sk, \userinfo, \accttodelete, \accttotransfer, \anset_1, \anset_2)$: 
    Διαγραφή ενός λογαριασμού μηδενικού υπολοίπου $\accttodelete$ από το σύνολο UTXO από τον ιδιοκτήτη του $\sk$ και προσθήκη των πληροφοριών ελέγχου του $(\varout, \varin)$ σε έναν άλλο λογαριασμό $\accttotransfer$ που μοιράζεται το ίδιο $\sk$. Τα σύνολα ανωνυμίας $\anset_1, \anset_2$ περιλαμβάνονται για την απόκρυψη των $\accttotransfer$ και $\userinfo$, αντίστοιχα.

    \item $0/1 \gets \vertxn(\txn, \state)$: 
    Είναι ένας δημόσιος αλγόριθμος επαλήθευσης που ελέγχει την εγκυρότητα μιας συναλλαγής $\txn$ δεδομένης της τρέχουσας κατάστασης $\state$ και εξάγει $1$ αν και μόνο αν είναι έγκυρη. 
    
    \item $\state' \gets \applytxn (\txn, \state)$:
    Χρησιμοποιείται για την εφαρμογή στην τρέχουσα κατάσταση μιας συναλλαγής $\txn$, μετά την επαλήθευσή της.
    
    \item $\auditinfo = (\pi, {\numaccs}_1, \{ \acct_{1i} \}_{i=1}^{\numaccs_1}, {\numaccs}_2, \{ \acct_{2i} \}_{i=1}^{\numaccs_2}) $ \\$\gets \audit(\sk,\inpk \snap_{1}, \snap_{2}, (f, \aux))$: 
    Χρησιμοποιείται από έναν χρήστη με ιδιωτικό κλειδί $\sk$ και αρχικό δημόσιο κλειδί $\inpk$ για τη δημιουργία μιας απόδειξης $\pi$ για τη συμμόρφωση με την πολιτική $f$, σχετικά με μια συγκεκριμένη χρονική περίοδο που ορίζεται από δύο στιγμιότυπα blockchain $\snap_1, \snap_2$. 
    Η μεταβλητή $\aux$ περιέχει τις βοηθητικές πληροφορίες που απαιτούνται για την πολιτική. 
    
    \item $0/1 \gets \vera(\inpk, \snap_1, \snap_2, (f,\aux), \auditinfo)$. 
    Χρησιμοποιείται από την Αρχή Ελέγχου για να ελέγξει αν ο χρήστης με το αρχικό δημόσιο κλειδί $\inpk$ συμμορφώνεται με την πολιτική $f$.

    Για την υλοποίηση των παραπάνω αλγορίθμων ο αναγνώστης παραπέμπεται στο αντίστοιχο κεφάλαιο της αγγλικής έκδοσης.

    \subsubsection{Μοντέλο Ασφάλειας}
    Ένα ανώνυμο σύστημα πληρωμών θα πρέπει να παρέχει ανωνυμία και πρόληψη έναντι κλοπής (theft prevention).
    Η ανωνυμία προϋποθέτει ότι ένας παρατηρητής του συστήματος δεν μπορεί να βρει τις ταυτότητες των αποστολέων και των παραληπτών μιας συναλλαγής, εάν δεν κατέχει το ιδιωτικό κλειδί του αποστολέα, και ότι ακόμη και ο παραλήπτης μιας συναλλαγής δεν μπορεί να γνωρίζει τον αποστολέα. Η πρόληψη κλοπής σημαίνει ότι οι χρήστες μπορούν να μεταφέρουν χρήματα μόνο από λογαριασμούς που τους ανήκουν. Επιπλέον, ένα ελεγχόμενο σύστημα πληρωμών απαιτεί την ιδιότητα ασφαλείας της ορθότητας του ελέγχου, η οποία σημαίνει ότι δεν μπορεί να υπάρξει ένας επιτυχώς επαληθευμένος έλεγχος που παράγεται από έναν μη συμμορφούμενο χρήστη.

    Στην εργασία κατασκευάζουμε ένα αυστηρό μοντέλο ασφάλειας που περιγράφει το AQQUA και βάση του οποίου αποδεικνείουμε ότι το AQQUA πληρεί τις παραπάνω ιδιότητες.


\end{itemize}

