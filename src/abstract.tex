\chapter*{Περίληψη}
\addcontentsline{toc}{chapter}{Περίληψη}
Η παρούσα διπλωματική εργασία είναι μια μελέτη των ιδιοτήτων ιδιωτικότητας και ελέγχου σε αποκεντρωμένα συστήματα πληρωμών.
Προτείνουμε το AQQUA: ένα ψηφιακό σύστημα πληρωμών που συνδυάζει την ιδιωτικότητα και την δυνατότητα ελέγχου.
Το AQQUA επεκτείνει το Quisquis, προσθέτοντας δύο αρχές: μία για την εγγραφή και μία για τον έλεγχο.
Αυτές οι αρχές δεν παρεμβαίνουν στην καθημερινή επεξεργασία των συναλλαγών- κατά συνέπεια, δεν διαταράσσεται η αποκεντρωμένη φύση του κρυπτονομίσματος. 
Η κατασκευή μας βασίζεται σε λογαριασμούς. 
Οι λογαριασμοί αποτελούνται από ένα updatable public key το οποίο λειτουργεί ως κρυπτογραφικά ασυσχέτιστο ψευδώνυμο,
και δεσμεύσεις για το υπόλοιπο, το συνολικό ποσό των νομισμάτων που δαπανήθηκαν και το συνολικό ποσό των νομισμάτων που ελήφθησαν.
Για να συμμετάσχει στο σύστημα, ο χρήστης δημιουργεί έναν αρχικό λογαριασμό στην αρχή εγγραφής. 
Για την προστασία της ιδιωτικότητάς του, κάθε φορά που θέλει να πραγματοποιήσει συναλλαγές δημιουργεί νέους λογαριασμούς που δεν μπορούν να συνδεθούν με τον χρήστη, ενημερώνοντας το δημόσιο κλειδί του και τον συνολικό αριθμό των λογαριασμών που κατέχει (που διατηρούνται σε δεσμευμένη μορφή). 
Η αρχή ελέγχου μπορεί να ζητήσει τον έλεγχο των χρηστών. Ο χρήστης πρέπει να αποδείξει με μηδενική γνώση ότι όλοι οι λογαριασμοί του συμμορφώνονται με συγκεκριμένες πολιτικές.
Ορίζουμε επίσημα ένα μοντέλο ασφάλειας για τις ιδιότητες που πρέπει να διαθέτει ένα ιδιωτικό και ελεγχόμενο ψηφιακό σύστημα πληρωμών και αναλύουμε την ασφάλεια του AQQUA σε σχέση με αυτό. 
\vspace{20ex}

\newlength\mylen
\settowidth\mylen{lastposupdate} % longest word in first column

\begin{table}[ht]
    \centering
    \begin{tabular}{p{1.5\mylen} p{\linewidth - 1.5\mylen}}
      \textbf{Λέξεις κλειδιά}: & ψηφιακά συστήματα πληρωμών, κρυπτονομίσματα, ιδιωτικότητα, δυνατότητα ελέγχου, updatable public keys. \\
    \end{tabular}
\end{table}


\newpage
\hspace{10pt}

\newpage
\chapter*{Abstract}
\addcontentsline{toc}{chapter}{Abstract}
This thesis is a study of privacy and auditability properties in decentralized payment systems.
We propose AQQUA: a digital payment system that combines auditability and privacy.
AQQUA extends Quisquis, by adding two authorities; one for registration and one for auditing.
These authorities do not intervene in the everyday transaction processing; as a consequence the decentralized nature of the cryptocurrency is not disturbed. 
Our construction is account-based. 
The accounts consist of an updatable public key which functions as a cryptographically unlinkable pseudonym,
and commitments to the balance, the total amount of coins spent and the total amount of coins received.
In order to participate in the system the user creates an initial account with the registration authority. 
To protect their privacy, whenever they want to transact they create unlinkable new accounts by updating their public key and the total number of accounts they own (maintained in committed form). 
The audit authority may request an audit at will. The user must prove in zero-knowledge that all their accounts are compliant to  specific policies.
We formally define a security model for the properties that a private and auditable digital payment system should possess and analyze the security of AQQUA in relation to it. 
\vspace{20ex}

% \settowidth\mylen{lastposupdate} % longest word in first column

\begin{table}[ht]
    \centering
    \begin{tabular}{p{\mylen} p{\linewidth - \mylen}}
      \textbf{Key words}: & digital payment systems, cryptocurrencies, privacy, auditability, updatable public keys. \\
    \end{tabular}
\end{table}

\newpage
\hspace{10pt}

\newpage
\chapter*{Ευχαριστίες}
\addcontentsline{toc}{chapter}{Ευχαριστίες}