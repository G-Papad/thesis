\section{Centralized Authority}

A common way to implement auditability in private systems is to introduce a centralized authority or group of authorities (multi-party computation). Such authority can either be an external designated auditor or can enforce the internal policy rules in each transaction (accountability). 

According to this approach, users embed auxiliary information in the transactions, which is encrypted under the public key of a designated trusted auditor (\autoref{fig:cent-aud}). Thus, the users' data remains private to the rest of the system's participants, except for the central authority, which can decrypt the auxiliary information at any time without the users' consent. 

\begin{figure}
    \centering
    \includegraphics[width=0.9\textwidth]{images/privacy/Auditability in blockchain - Centralized.png}
    \caption{Auditability - Centralized Authority}
    \label{fig:cent-aud}
\end{figure}


This method can be a trivial solution for adding auditability to privacy-preserving systems. However, all data is collected by a single centralized authority, which accumulates excessive power. This fact can have a negative impact on user privacy.

An example that implements this approach is presented below:

\subsection{Zcash extension}

\cite{GGM16} is an extension of Zerocash \cite{Zerocash} to add privacy-preserving policy enforcement mechanisms that ensure regulatory compliance. It provides a variety of auditability features including regulatory closure, transaction spending limits, and accountable selective user tracing. To achieve this, auxiliary information is added to each Zerocoin (counter, regulatory type) as well as algorithms-policies that are executed each time a coin is spent. 

They introduce a new basic building block in Zerocash, \emph{Counters}, which is added to the coin data. Counters can store cumulative information either for a specific physical user or for Zerocash addresses. In the first case, counters should be tied to a specific (unique) identity and issued by a trusted third party. In the second case, a user may have many addresses within the system, and an authority should know the link between the user's real-world identity and the set of internal addresses. To preserve anonymity, these counters must be entered using the same method as coin commitments, which involves proving their inclusion in a Merkle tree.

Counters are important for enforcing spending limits and tax policies. \emph{Spending Limit Policy} enforces that no transaction with a transfer or counter value over a specified limit is valid unless it is signed by an authority. This provides a form of accountability as it can preemptively review transactions before they are entered into the ledger. On the other hand, \emph{Tax} belongs to the auditability type policy. All outputs sent to other parties are summed up and a percentage of that amount is added to a user's tax counter. After that, the authority is responsible for auditing the users for their tax compliance. 
 
The second piece of information embedded in each coin is the regulatory type. This data is used to achieve regulatory closure, meaning that the system can provide enough information to guarantee the continuation of another regulatory framework that can be enforced with or without zero-knowledge. This policy ensures that all input and output coins in a transaction have the same regulatory type, and thus an adversary cannot cause policies to operate on the wrong data.

Finally, \cite{GGM16} offers the ability of accountable coin tracing. The coins can also contain tracing information encrypted under a unique key held by the tracing authority. The authority could then trace these coins, along with all subsequent coins resulting from transactions involving the original coins, without requiring any interaction with the users. However, there is a mechanism that allows a user to find out whether or not they have been traced. If a user is being traced, the key given by the authority will be a randomized version of their public key, otherwise it will be a randomized version of a null key. Users cannot tell which key they have received at any given time without being aware of the randomness. However, if the authority later reveals the randomness, users can definitively determine whether or not they have been traced. 

Although \cite{GGM16} allows the implementation of a wide variety of regulatory policies, it suffers from both efficiency and privacy limitations. Zerocash is already an computationally intensive and storage expensive system since it has a monotonically growing list of UTXO \cite{SoKPrivacyPreservingComputing}.  The addition of auxiliary information and the requirement that transactions be validated by an authority before posting on-chain adds additional communication and computation costs. In addition, the designated authority has too much power with respect to user privacy, since it can learn any transaction information and deanonymize the user. Furthermore, there is no mechanism to prevent censorship of certain transactions by the authority \cite{SokAuditability}.
% \input{src/audit_schemes/prcash.tex}
% \input{src/audit_schemes/accdet.tex}


\section{General Auditor}

To avoid collecting all information at a centralized authority (or group of authorities), a second approach has been proposed. This is an interactive protocol between the user being audited and the auditor. In this case, the auditor, which can be \textit{any} auditing authority, can ask specified questions derived from the system's policies. Users answer these questions with zero-knowledge proofs based on data stored on chain (\autoref{fig:gen-aud}).

This protocol implies the consent and cooperation of the audited user. However, this requirement cannot be exploited by non-compliant users, since refusal to cooperate with authorities can be considered equivalent to a failed audit.

\begin{figure}
    \centering
    \includegraphics[width=0.9\textwidth]{images/privacy/Auditability in blockchain - General Auditor.png}
    \caption{Auditability - General Auditor}
    \label{fig:gen-aud}
\end{figure}

Below are some typical examples of auditable private decentralized systems that implement a general auditor:
\subsection{zkLedger}
zkLedger \cite{zkLedger} is a permissioned, fully-private payment system that supports auditing by a general auditor. It succeeds to provide strong transaction privacy properties, such as hiding the transfered amounts, the participants as well as the link between the transactions (transaction graph), while allowing an auditor to compute provably correct functions over the on-chain data. Audit procedure takes place as an interactive protocol between the users and an auditor, where the latter queries the former about its contents on the ledger. 

Since an auditor cannot distinguish the participants of a transaction, zkLedger should ensure that during auditing a user cannot leave out transactions that participated, in order to be able to receive reliable answer to its queries.

The solution, that zkLedger proposed, to this problem relies on its unique ledger. That is, a table where transactions correspond to rows, and Users (Banks) correspond to columns. Each transaction include information for all other users, even for those who do not participate. To hide the transfered amounts as well as the assets each user holds each entry in a transaction are contains a commitment to a value that is debited or credited to the user. All entries for the non-participants are a commited value to 0 (\autoref{fig:zkLedger-table}). Due to the hiding property of the commitments an adversary cannot distinguish between a zero and a non-zero commited value.

\subsubsection{Transactions}
The transactions in zkLedger are publicly verifiable and must satisfy the following invariants:
\begin{itemize}
    \item Transactions conserves assets, meaning a transfer transaction cannot create or destroy assets.
    \item The spending user gives concent to the transfer and actually own enough assets to execute the transaction.
    \item After each transactions all users have enough information to open their commitments for the audit functionality.
\end{itemize}

Therefore, transactions are formed via a combination of commitments and the zero-knowledge proofs and have the following form $\txn = (cm_i, Token_i, \pi^B, \pi^A, \pi^C)$:
\begin{itemize}
    \item \textbf{Commitment ($cm_i$)}: $(g^{v_i}h^{r_i})$ a Pedersen commitment to the transfered value.
    \item \textbf{Audit token ($Token_i$)}: $(pk_i)^r_i$ used in order to enable user to create reliable answers to the audits without knowing the randomness used in the commitment.
    \item \textbf{Proof of balance ($\pi^B$)}: a zero-knowledge proof asserting the preservation of balance in each transaction $\sum_{k=1}^n v_k = 0$.
    \item \textbf{Proof of assets ($\pi^A$)}: a zero-knowledge proof asserting that the sender has the assets to transfer. In zkLedger a column in the ledger represents all the assets the corresponding user has received or spent. So $\pi^A$ is proven by creating a commitment to the sum of values for the asset in its column, including the current transaction. If the sum is greater than or equal to 0, then the user has the right to transfer the amount. 
    \item \textbf{Proof of consistency ($\pi^C$)}: a zero-knowledge proof asserting that a malicious user cannot add  data to the ledger that would prevent another user to be able to open their commitments.
\end{itemize}

Finally, whenever a transactions happens, a new row is added to the ledger.

\begin{figure}
    \centering
    \includegraphics[width=0.9\textwidth]{images/zkLedger/zkledger.png}
    \caption{zkLedger Ledger - naive transaction example}
    \label{fig:zkLedger-table}
\end{figure}

\subsubsection{Auditing}

The Auditor has access to the ledger and interact with the users to calculate functions on their private data. This functions allow the auditor to issues queries that includes information about ratios of holdings, sums, average, variance, outliers, changes over time. The account holder reveal only the value hidden in commitment necessary for the question, without leaking any other information. However, frequent auditing is possible to reveal more of a transaction contents.

\autoref{fig:zkLedger-audit} illustrate an example. An auditor can ask a user "How many euros do you hold at time $t$?". The user responds with an answer along with the validity zero-knowledge proof. The auditor can multiply all the commitments on the ledger for the audited user and can verify if the proof and the answer are valid. Since each column of the ledger represents all the assets that the corresponding user has received or spent, the auditor can be sure that user could not hide any of their transaction during the audit.

\begin{figure}
    \centering
    \includegraphics[width=0.9\textwidth]{images/zkLedger/zkledger-audit.png}
    \caption{zkLedger - audit functionality example}
    \label{fig:zkLedger-audit}
\end{figure}

\subsubsection{Setbacks}
Although zkLedger combines full privacy with auditability and offers a wide variety of audit functionalities, suffers from very limited scalability. There are two significant costs that grow with the number of users. The verification of transactions which increases with the number of the users as well as The sequential steps to create transactions increase linearly. 

Considering the verification, zkLedger requires each transaction to include commitments and zero-knowledge proofs for all users. This fact, combined with the ever-increasing ledger that grows with every transaction, results in large computational and storage costs. To overcome this problem an improvement of zkLedger protocol has been proposed called miniLedger \cite{MiniLedger}.

Concerning the second setback, in order a user to be able to produce a new transaction (for exmample transaction $n$) they must use the state of the ledger right before, meaning the must known the $n-1$ transaction. Therefore, multiple users cannot produce different transactions in parallel, since concurrent transactions always have conflicts.
% \input{src/audit_schemes/miniledger.tex}
\subsection{PGC}

PGC \cite{PGC} is a standalone auditable confidential payment system. In this work they trade anonymity for highly efficient auditing only dependent to the number of past user transactions. Privacy is offered in terms of confidentiality and use pseudonymity as a feature assuming that auditors can link the account addresses with real identities. It proposes two kinds of auditing mechanism: (i) regulation compliance that is achieved through three audit functions, namely transaction limits, tax payment and selective value disclosure. (ii) supervision (tracing functionality) that is achieved through including the necessary auxilary information in the transaction structure. The cryptographic techniques that are used for its implementation is a variant of El Gamal encryption and zk-proofs composed of $\Sigma$-protocols.

\subsubsection{Entities}
More specifically, the system consists of the following entities:
\begin{itemize}
    \item Users: Individuals that transact with each other and may control several accounts within the system.

    \item Validators: Validators are responsible for checking the validity of proposed transactions within the system. They ensure that transactions meet the required criteria before being included in the blockchain.
    
    \item Regulators: Regulators interact with involved users to verify if a set of transactions complies with system's policies, without holding any secret information.
    
    \item Supervisors: Supervisors have access to a global trapdoor, which allows them to monitor and trace transactions without interacting with the involved users. 
\end{itemize}

\subsubsection{Accounts}
In order to be able to interact with a system a user creates an account. Each account is associated with a secret key $\sk$, a public key $\pk$, which represent the pseudonym of the user within the system, an encoded balance $\tilde{C}$ , and an incremental serial number $sn$ used to prevent replay attacks ($ \acct = (\sk, \pk, \tilde{C}, sn)$). The balance is encrypted in order privacy to be achieved. Only the owner of $\sk$ can decrypt and learn the value but all users should be able to change the encrypted value. PGC implement this through homomorphic encryption.

\subsubsection{Transactions}
Users can use the accounts to transact within the system. A transaction take place between two participants and need as input the secret key of the sender $\sk$, the transacted value $v$ and the public keys of both sender and receiver ($\pk_s, \pk_r)$. Given the input the algorithm produces $(C_s, C_r)$ that are encoded transferred value $v$ under the public keys of sender and receiver $(\pk_s, \pk_r)$.

The legality of the transaction is proved through the creation of a a zk-proof $\pi_{legal}$. $\pi_{legal}$ consists of the following proofs: (i) $\pi_{equal}$: $C_s, C_r$ contains encryption of the same value $v$ (ii) $\pi_{right}$: the transfer amount $v$ is within the allowed limits (iii) $\pi_{solvent}$: the sender has enough balance.

Finally, to authenticate that the sender is the owner of the corresponding account the algorithm sign the $(sn, memo = (\pk_s, \pk_r, C_s, C_r), \pi_{legal})$ with the secret key $\sk$ producing the signature $\sigma$.

The final transaction is $\txn = (sn, memo, \pi_{legal}, \sigma)$.

In order to support supervision under a specified authority with a known public key $\pk_a$ the transaction can be extended with an encryption of transacted value $C_a$ under the $\pk_a$. Then the supervisor can inspect any confidential transaction by decrypting $C_a$ using $\sk_a$.

\begin{figure}[h]
    \includegraphics[width=\textwidth]{images/pgc/pgc_txn.png}
    \centering
    \caption{Data structure of transaction in PGC}
\end{figure}


\subsubsection{Policies and Audit}
In PGC policies are represented as predicates $f$ over a public key $\pk$ and related transactions $\{\txn_i\}_{i=1}^n$ in which $\pk$ participates either as sender or as receiver. Let $v_i$ the transferred amount in $\txn_i$. They implement the following policies over the values $v_i$:
\begin{itemize}
    \item Limit policy $f_{limit}(\pk, \{\txn\}_{i=1}^n)$: Checks that $\sum_i^n {v_i} \leq a_{max}$, where $a_{max}$ is an upper bound depending on application. The prover can be either the sender or the receiver of $\txn$. This policy is a mechanism used for anti-laundering money.
    \item Tax policy $f_{tax}(\pk, \txn_1, \txn_2)$: Let $\pk$ be recipient in $\txn_1$ and sender in $\txn_2$. Let $v_1,v_2$ be the transfer amounts in each transaction. The policy checks if $v_1/v_2 = \rho$, where $\rho$ is a rate depending on application. The auditor can use this policy to ensure that user paid appropriate tax.   
    \item Open policy $f_{open}(\pk, \txn)$: Checks that the underlying transferred amount $v$ is equal to a $v^*$ ($v=v^*$), where $v^*$ is an application-dependent value. The prover can be either the sender or the receiver of $\txn$. The auditor can use this policy to enforce selective-disclosure.
\end{itemize}

\begin{figure}[h]
    \includegraphics[width=\textwidth]{images/pgc/pgc_policies.png}
    \centering
    \caption{Policies in PGC}
\end{figure}


In order the auditing to be efficient in PGC it is executed with the aid from the auditee. In particular, it needs the consent of the auditee who creates a zk-proof that proves that they are compliant with the specified policy. The zk-proofs are implemented by using bulletproofs and $\Sigma$-protocols.

% \subsubsection{Implementation}
% As mentioned PGC is composed of a homomorphic encryption scheme, a signature scheme and zk-proofs. In order to implement the first an ISE (Integrated Signature and Encryption) scheme is used. This ISE scheme is instantiated by combining a version of El Gamal encryption, Twisted El Gamal \autoref{TwistedElGamal}, and Schnorr signatures \autoref{Schnorr}, due to the fact that this two cryptographic primitives share the same $\setup, \kgen$ algorithms.In \cite{PGC} it is proved that the obtained ISE scheme is jointly secure if the the twisted ElGamal is IND-CPA secure and the Schnorr signature is EUF-CMA secure.

% Using this implementation the NIZK protocol for the proof system can be created for the validity of the transactions and for the system policies.

% As mentioned in the transactions, $L_{legal}$ can be decomposed in $L_{equal} \land L_{right} \land L_{open}$. 
% \begin{itemize}
%     \item \textbf{NIZK for $L_{equal}$}:\\
%         According to implementation $L_{equal}$ can be written as:
%         \begin{equation*}
%             \{(\pk_1, X_1, Y_1, \pk_2, X_2, Y_2) | \exists r, v \text{ s.t. } X_i = \pk_i^{r} \land Y_i = g^{r} h^v \text{ for } i=1,2 \}
%         \end{equation*}
%         In Twisted ElGamal the randomness $r$ can be safely reused in the 1-plaintext/2-recipient setting. % //TODO: prove this in the crypto section

%         \textbf{$\Sigma$-protocol $\Sigma_{equal}$}:
%         \begin{center}
%             \begin{tabular}{ |l c l| }
%              \hline
%              & ($\pk_1, \pk_2, X_1, X_2, Y$) & \\   
%              \textbf{Prover}($r,v$) &  & \textbf{Verifier} \\
%              $a,b \sample \mathbb{Z}_p$ & & \\
%              $A_1 \gets \pk_1^a, A_2 \gets \pk_2^a$ & $\xlongrightarrow{A,B}$ & \\ 
%              & $\xlongleftarrow{e}$ & $e \sample \mathbb{Z}_p$ \\
%              $z_1 \leftarrow er+a$ &  & \\
%              $z_2 \leftarrow ev+b$ & $\xlongrightarrow{z_1, z_2}$ & Check\\
%              & & $\pk^{z_1} =  AX^e$ \\
%              & & $g^{z_1} h^{z_2} = BY^e$ \\  
%              \hline
%             \end{tabular}
%         \end{center}
%     \item \textbf{NIZK for $L_{right}$}
%     \item \textbf{NIZK for $L_{open}$}
% \end{itemize}

\subsubsection{Setback}
Although PGC does not suffer from the scalability issues of zkLedger, it is not fully private. As mentioned before, PGC offers only confidentiality when it comes to privacy and trades anonymity in order to implement efficient auditability.