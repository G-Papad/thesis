\subsection{Zcash extension}

\cite{GGM16} is an extension of Zerocash \cite{Zerocash} to add privacy-preserving policy enforcement mechanisms that ensure regulatory compliance. It provides a variety of auditability features including regulatory closure, transaction spending limits, and accountable selective user tracing. To achieve this, auxiliary information is added to each Zerocoin (counter, regulatory type) as well as algorithms-policies that are executed each time a coin is spent. 

They introduce a new basic building block in Zerocash, \emph{Counters}, which is added to the coin data. Counters can store cumulative information either for a specific physical user or for Zerocash addresses. In the first case, counters should be tied to a specific (unique) identity and issued by a trusted third party. In the second case, a user may have many addresses within the system, and an authority should know the link between the user's real-world identity and the set of internal addresses. To preserve anonymity, these counters must be entered using the same method as coin commitments, which involves proving their inclusion in a Merkle tree.

Counters are important for enforcing spending limits and tax policies. \emph{Spending Limit Policy} enforces that no transaction with a transfer or counter value over a specified limit is valid unless it is signed by an authority. This provides a form of accountability as it can preemptively review transactions before they are entered into the ledger. On the other hand, \emph{Tax} belongs to the auditability type policy. All outputs sent to other parties are summed up and a percentage of that amount is added to a user's tax counter. After that, the authority is responsible for auditing the users for their tax compliance. 
 
The second piece of information embedded in each coin is the regulatory type. This data is used to achieve regulatory closure, meaning that the system can provide enough information to guarantee the continuation of another regulatory framework that can be enforced with or without zero-knowledge. This policy ensures that all input and output coins in a transaction have the same regulatory type, and thus an adversary cannot cause policies to operate on the wrong data.

Finally, \cite{GGM16} offers the ability of accountable coin tracing. The coins can also contain tracing information encrypted under a unique key held by the tracing authority. The authority could then trace these coins, along with all subsequent coins resulting from transactions involving the original coins, without requiring any interaction with the users. However, there is a mechanism that allows a user to find out whether or not they have been traced. If a user is being traced, the key given by the authority will be a randomized version of their public key, otherwise it will be a randomized version of a null key. Users cannot tell which key they have received at any given time without being aware of the randomness. However, if the authority later reveals the randomness, users can definitively determine whether or not they have been traced. 

Although \cite{GGM16} allows the implementation of a wide variety of regulatory policies, it suffers from both efficiency and privacy limitations. Zerocash is already an computationally intensive and storage expensive system since it has a monotonically growing list of UTXO \cite{SoKPrivacyPreservingComputing}.  The addition of auxiliary information and the requirement that transactions be validated by an authority before posting on-chain adds additional communication and computation costs. In addition, the designated authority has too much power with respect to user privacy, since it can learn any transaction information and deanonymize the user. Furthermore, there is no mechanism to prevent censorship of certain transactions by the authority \cite{SokAuditability}.