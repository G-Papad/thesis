\subsection{Zcash extension}

\cite{GGM16} is an extension of the Zerocash \cite{Zerocash} in order to add privacy-preserving policy-enforcement mechanisms that guarantee regulatory compliance. It offers a variety of auditability functionalities including regulatory closure, transactions spending limit and accountable selective user tracing. To achieve this auxiliary information is added to each zerocoin (Counters, regulatory type) as well as algorithms-policies that are executed each time a coin is spent. 

They introduce a new basic building block in Zerocash, \emph{Counters}, which is added in coin data. Counters can store cumulative information either for a specific physical user or for Zerocash addresses. In the first case counters should be bound to a specific (unique) identity and issued by a trusted third party. On the second case, a user can own lots of addresses inside the system and an authority should know the link between the real-world identity of the user and the set of inside addresses. To preserve anonymity, these counters must be entered using the same method as coin commitments, which involves proving their inclusion in a Merkle tree.

Counters are important for the implementation of spending limits and tax policies. \emph{Spending limit policy} enforce that no transaction with transfered or counter's value over a specified limit is valid unless is singed by an authority. This represents a form of accountability as it can preemptively review transactions prior to their entry into the ledger. On the other hand, \emph{Tax} belongs to auditability type policy. All outputs being sent to other parties are summed and a percentage of that amount is added to a  user's tax counter. Afterwards, the authority is responsible to audit the users for their tax compliance. 
 
The second embeded information to each coin is regulatory type. This data is used in order to achieve regulatory closure, meaning that the system can provide enough information to guarantee the continuation of another regulatory framework, which could be enforced with or without zero-knowledge. This policy ensures that all inputs and outputs coins in a transaction have the same regulatory type and hence an adversary cannot cause policies to operate over the wrong data.

Finally, \cite{GGM16} offers the ability of accountable coin tracing. The coins can also include tracing information encrypted under a unique key held by the tracing authority. Subsequently, the authority could trace those coins, along with all subsequent coins resulting from transactions involving the original coins, without requiring any interaction with the users. However, there is a mechanism that allows a user to figure if they have been traced or not. If a user is being traced the key that the authority has given will be a randomized version of its public key, else it will be a randomized version of a null key. Users cannot distingouish which key they receive at any given time without awareness of the randomness. Nonetheless, if the authority discloses the randomness later on, users can definitively determine whether they were traced or not. 

Altough \cite{GGM16} allows the implementation of a wide variety of regulatory policies, it suffers from both efficiency and privacy limitations. Zerocash is already an computationally intensive and storage expensive system since it has a monotonically growing list of UTXO \cite{SoKPrivacyPreservingComputing}. Adding auxiliary information and the requirement of the transactions being validated by an authority before posting on-chain adds extra communication and computation costs. In addition, concernig the privacy of the user the designated authority has too much power due to the fact that can learn any transaction information and deanonymize the user. In addition, there does not exists any mechanism to prevent cencorship to certain transactions from the authority \cite{SokAuditability}.