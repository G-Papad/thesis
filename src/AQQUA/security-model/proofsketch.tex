% \subsection{Proofs of security}

\subsubsection{Proof of anonymity}
%The complete proof of anonymity is provided in the Appendix; however, we'll outline the main intuition here.
Intuitively, we argue that any PPT adversary $\adv$ capable of distinguishing between $\txn_0, \txn_1$ in the anonymity game (find if $b'=b$) can be used to break either the indistinguishability of UPK scheme, the hiding property of commitment scheme, or the zero-knowledge property of the NIZK proofs. 

Transactions consist of $\inputs, \outputs$, and a zk-proof $\pi$ (and if it is $\createAcct$ or $\deleteAcct$ a newly created account $\acct$). One way $\adv$ could determine $b$ is based  on $\pi$, but that violates the zero-knowledge property of the NIZK proofs. 
Another way that $\adv$ could determine $b$ is to distinguish between $\txn_0, \txn_1$ through the $\outputs$ sets of each $\txn$. The only differences in the two  $\outputs$ sets $\txn_0.{\outputs}, \txn_1.{\outputs}$ are the accounts which are used in $\pset$ and in $\anset$ as well as the amount $v$ used to increase/decrease the variables in the accounts of $\pset$.
However, since both the accounts' amounts and transferred value $v$ are presented in a committed form, if $\adv$ can determine $b$ based on the different values $v_0, v_1$ then the hiding property of the commitment scheme is violated.
In addition, since all the accounts participating in the transaction are updated and randomly permuted, 
% both those in $\pset$ and in $\anset$ (and the newly created account is an update of a former public key), 
 $\adv$ cannot use $\pset_0, \anset_0,\pset_1, \anset_1$ to distinguish the two transactions without violating the indistinguishability property of UPK scheme.

 % \section{Full proof of anonymity}

% Before we give the proof of anonymity, we first recall a  definition for indistinguishability of UPK scheme \cite{fauzi2019quisquis}.
% \begin{definition}\label{def:UPKindistinguishability} 
%     The \emph{advantage} of the adversary in winning the indistinguishability game is defined as:
%     $
%         \advantage{ind}{\adv} = \mid \Pr[\exper^{ind}_{\adv}((\secpar))=1] - \dfrac{1}{2} \mid
%     $

%     A DPS satisfies \emph{indistinguishability} if for every PPT adversary $\adv$, $\advantage{ind}{\adv}$ is negligible in  $\secpar$.
% \end{definition}
% \begin{game}[htbp]
    % \DontPrintSemicolon
    % \SetAlgoLined
%     \caption{Indistinguishability game $\exper^{ind}_{\adv}(\secpar)$}
%     \label{alg:UPKindistinguishability-game}
    % \SetKwInOut{Input}{Input}
    % \SetKwInOut{Output}{Output}
%     \Input{$\secpar$}
%     \Output{$\{0,1\}$}
%     \BlankLine
%     $b\gets \{0, 1\}$ \;
%     $(\pk^*, \sk^*) \gets \kgen()$\;
%     $r \sample \randspace$\;
%     $\pk_0 \gets \upd(\pk^*;r)$\;
%     $(\pk_1, \sk_1) \gets \kgen()$\;
%     $b^\prime \gets \adv(\pk^*, \pk_b)$\;
%     \Return{$(b = b^\prime)$}
% \end{game}
% Note that in indistinguishability game the challenger can update many times the $\pk^*$ before creating $\pk_0$ due to the fact that even with more updates the $pk_0$ can be described as an update of $\pk^*$ with a different randomness.

% \begin{lemma}
%     The constructed UPK scheme satisfies \ref{def:UPKindistinguishability} if the DDH assumption holds in $(\group, g, p)$.
% \end{lemma}
% Proof of this lemma can be found on \cite{fauzi2019quisquis}.

\begin{theorem}
    AQQUA satisfies anonymity, as defined in \autoref{def:anonymity}
\end{theorem}
\begin{proof}
We prove the theorem using a sequence of 14 hybrid games, as follows. Hybrid 0 and Hybrid 7 are the anonymity game for $b=0, b=1$ respectively. Each of the rest hybrids differs in oracles' functionalities in a way that the successive hybrids are indistinguishable from the view of the adversary. We use these hybrids to prove that the adversary cannot distinguish anonymity game for $b=0$ and anonymity game with $b=1$. \\
\textbf{Hybrid 0.} The anonymity game for $b=0$. \\
% 
\textbf{Hybrid 1.} Same as Hybrid 0, but here we run the NIZK extractor on each transaction generated by the adversary. That means, when $\adv$ runs the $\applytxnOracle(\txn)$ Oracle, the Oracle verifies $\txn$ by running $\vertxn(\txn, \state)$ 
% or $\vercreateAcct(\txn, \state)$ 
%or $\verdeleteAcct(\txn, \state)$ 
depending on the transaction $\txn$ and if it is successful the oracle runs $\state^\prime \gets \applytxn(\state, \txn)$, as well as uses the NIZK extractor to extract the witness used to generate $\txn$, including $\sk$. \\
% 
\textbf{Hybrid 2.} Same as Hybrid 1, but here the zero-knowledge arguments of the each transaction is replaced with the output of the corresponding simulator of the zero-knowledge property of NIZK. In order to achieve this we change the following oracles' functionality:
\begin{itemize} 
    \item when $\adv$ or the challenger creates $\txn$ through the $\transOracle(\sset, \rset, \vv{\v_\sset}, \vv{\v_\rset}, \anset)$ Oracle, the Oracle runs $\txn \gets \trans(\sk,\sset, \rset, \vv{\v_\sset}, \vv{\v_\rset}, \anset)$, but replaces the zero-knowledge arguments in $\txn$ with a simulated argument.\\
    
    \item when $\adv$ or the challenger creates $\txn$ through the $\createAcctOracle(\userinfo, \anset)$ Oracle, the Oracle runs $\txn \gets \createAcct(\userinfo, \anset)$, but replaces the zero-knowledge arguments in $\txn$ with a simulated argument.\\
    % \item when $\adv$ or the challenger creates $\txn$ through the $\deleteAcctOracle(\userinfo, \acct_c, \acct_d, \anset)$ Oracle, the Oracle runs $\txn \gets \deleteAcct(\sk, \acct_d, \acct_c, \anset)$, but replaces the zero-knowledge arguments in $\txn$ with a simulated argument.\\
\end{itemize}
% 
\textbf{Hybrid 3.} Same as Hybrid 2, but now the challenger replaces the potential senders' and receivers' accounts of the challenge transaction $\txn_0$ $(\acct_0, \acct_1, \acct_0^\prime, \acct_1^\prime)$, with new accounts that have a freshly created key pair ($\sk, \pk)$ derived from the output of the $\kgen()$. 
In order to achieve this we change the following oracles' functionality:
\begin{itemize} 
    \item when $\adv$ creates one of these accounts $\acct_i$ through the $\transOracle$ Oracle (these accounts are presented in $\txn.\outputs$), the Oracle runs $\txn \gets \trans(\sk,\sset, \rset, \vv{\v_\sset},$ \\ $ \vv{\v_\rset}, \anset), (\pk_i^\prime, \sk_i^\prime) \gets \kgen$ and then return $\txn^\prime$, where $\txn^\prime = \txn$ except that each $\acct_i \in \{\acct_0, \acct_1, \acct_0^\prime, \acct_1^\prime\}$ is replaced with $\acct_i^\prime = (\pk_i^\prime, \com_{\varbl i}, \com_{\varout i},$ \\ $ \com_{\varin i})$.\\
    
    \item when $\adv$ creates one of these accounts $\acct_i$ through the $\createAcctOracle$ Oracle, the Oracle runs $\txn \gets \createAcct(\userinfo, \anset), (\pk_i^\prime, \sk_i^\prime) \gets \kgen$ and then return $\txn^\prime$, where $\txn^\prime = \txn$ except that each $\acct_i \in \{\acct_0, \acct_1, \acct_0^\prime, \acct_1^\prime\}$ is replaced with 
    % $\acct_i^\prime = (\pk_i^\prime, \com_{\varbl i}, \com_{\varout i}, \com_{\varin i})$.\\
    $\acct_i^\prime = (\pk_i^\prime, \comm{0}, \comm{0}, \comm{0})$.\\
\end{itemize}
% \textbf{Hybrid 2 | 6.} Same as Hybrid 1 | Hybrid 5, but now the challenger replace the real participants of $\txn_0$, meaning the accounts $(\acct_0, \acct_1, \acct_0^\prime, \acct_1^\prime)$, with fresh accounts that have a newly created public key derived from the output of the $\kgen$ and commitments of the same value under the new $\pk$ and a different randomness. That is, when $\adv$ creates one of these accounts $\acct_i$ through the $\transOracle$ Oracle (when these accounts are presented in the outputs of a $\txn$), the Oracle runs $\txn \gets \trans(\sk,\sset, \rset, \vv{\v_\sset}, \vv{\v_\rset}, \anset), (\pk^\prime, \sk^\prime) \gets \kgen, \com_\varbl \gets \commit(\pk^\prime,\acct_i.\varbl), \com_\varout \gets \commit(\pk^\prime,\acct_i.\varout), \com_\varin \gets \commit(\pk^\prime,\acct_i.\varin)$ and then return $\txn^\prime$, where $\txn^\prime = \txn$ except that each $\acct_i \in \{\acct_0, \acct_1, \acct_0^\prime, \acct_1^\prime\}$ is replaced with $\acct^\prime = (\pk^\prime, \com_\varbl, \com_\varout, \com_\varin)$.\\
% 
\textbf{Hybrid 4.} Same as Hybrid 3, but here the challenger replaces also the commitments of the accounts $(\acct_0, \acct_1, \acct_0^\prime, \acct_1^\prime)$ with newly created commitments to the same values with different randomness. 
In order to achieve this we change the following oracles' functionality:
\begin{itemize} 
    \item when $\adv$ creates one of these accounts $\acct_i$ through the $\transOracle$ Oracle (these accounts are presented in $\txn.\outputs$), the Oracle runs $\txn \gets \trans(\sk,\sset, \rset, \vv{\v_\sset},$ \\ $ \vv{\v_\rset}, \anset), (r_1, r_2, r_3) \sample \randspace,
    \varbl_i \gets \opencom(\sk, \acct_i.\com_{\varbl}), \varout_i \gets \opencom(\sk, $ \\ $\acct_i.\com_{\varout}), \varin_i \gets \opencom(\sk, \acct_i.\com_{\varin}), 
    \com_\varbl^\prime \gets \commit(\pk^\prime, \varbl_i; r_1), $ \\ $\com_\varout^\prime \gets \commit(\pk^\prime,\varout_i;r_2), \com_\varin^\prime \gets \commit(\pk^\prime,\varin_i; r_3)$ and then return $\txn^\prime$, where $\txn^\prime = \txn$ except that each $\acct_i \in \{\acct_0, \acct_1, \acct_0^\prime, \acct_1^\prime\}$ is replaced with $\acct^\prime = (\pk, \com_\varbl^\prime, \com_\varout^\prime, \com_\varin^\prime)$. ($\pk = \pk^\prime$ as in the Hybrid 3). \\
    
    \item when $\adv$ creates one of these accounts $\acct_i$ through the $\createAcctOracle$ Oracle, the Oracle runs $\txn \gets \createAcct(\userinfo, \anset), (r_1, r_2, r_3) \sample \randspace,  \com_\varbl^\prime \gets \commit(\pk^\prime0; r_1), \com_\varout^\prime \gets \commit(\pk^\prime,0;r_2), \com_\varin^\prime \gets \commit(\pk^\prime,0; r_3)$ and then return $\txn^\prime$, where $\txn^\prime = \txn$ except that each $\acct_i \in \{\acct_0, \acct_1, \acct_0^\prime, \acct_1^\prime\}$ is replaced with $\acct^\prime = (\pk, \com_\varbl^\prime, \com_\varout^\prime, \com_\varin^\prime)$. ($\pk = \pk^\prime$ as in the Hybrid 3). \\
\end{itemize}
% 
\textbf{Hybrid 5.} Same as Hybrid 4, but here also the updated accounts of $(\acct_0, \acct_1, \acct_0^\prime, \acct_1^\prime)$ in the challenge $\txn.\outputs$ are replaced by accounts with freshly created public key $\pk^\prime$.\\
% 
\textbf{Hybrid 6.} Same as Hybrid 5, but here also the updated accounts of $(\acct_0, \acct_1, \acct_0^\prime, \acct_1^\prime)$ in the challenge $\txn.\outputs$ are replaced by accounts with freshly created commitments to the same value.\\

Afterwards, we create Hybrids 7-13 that are the same with Hybrids 0-6 with the difference that are made for the anonymity game with $b=1$.

Note that in Hybrid 6 and in  Hybrid 13 all accounts of the potential senders' and receivers' accounts of the challenge transaction $\txn_b$ (both in $\inputs$ and $\outputs$) are fresh accounts, where in $\outputs$ have been generated with values corresponding to the case $b=0$ | $b=1$.

Now we will prove that $\adv$ has negligible advantage of distinguish Hybrid 0 and Hybrid 7. \\
% //TODO: Prove each Lemma:
% Lemma: Hybrid 1 and Hybrid 2 are indistinguishable -> zk property \\
% Lemma: Hybrid 2 and Hybrid 3 are indistinguishable -> UPK  indistinguishability + key-anonymous of commitments\\
% Lemma: Hybrid 3 and Hybrid 4 are indistinguishable -> hiding of commitments \\
% Lemma: Hybrid 4 and Hybrid 5 are indistinguishable -> UPK  indistinguishability + key-anonymous of commitments\\
% Lemma: Hybrid 5 and Hybrid 6 are indistinguishable -> hiding of commitments \\
% Lemma: Hybrid 6 and Hybrid 13 are indistinguishable -> hiding of commitments \\
\begin{lemma}
    Hybrid 0 and Hybrid 1 are indistinguishable.
\end{lemma}
\begin{corollary}
    Hybrid 7 and Hybrid 8 are indistinguishable.
\end{corollary}
\begin{proof}
The adversary's view in the two hybrids' game are identical.
\end{proof}

\begin{lemma}
    Hybrid 1 and Hybrid 2 are indistinguishable.
\end{lemma}
\begin{corollary}
    Hybrid 8 and Hybrid 9 are indistinguishable.
\end{corollary}
\begin{proof}
    Let $\adv$ be an adversary that can distinguish Hybrid 1 and Hybrid 2 with advantage $\epsilon$. We construct an adversary $\badv$ that breaks the zero-knowledge property of the NIZK proof $\pi$ of transaction $\txn$ with probability $\epsilon$.
    
    Let $\pcalgostyle{O_{zk}(\cdot)}$ be an oracle that on input $(\txn.\inputs, \txn.\outputs)$ creates a valid zero-knowledge proof for the transaction. Then $\badv$ wins if they can decide wether $\pcalgostyle{O_{zk}(\cdot)}$ is a prover or simulator oracle.
    
    $\badv$ takes as input the $\pcalgostyle{O_{zk}(\cdot)}$ and runs as follows:
    \begin{enumerate}
        \item $\badv$ generates $\state \gets \setup(\secpar)$;
        \item When $\adv$ queries the $\transOracle(\sset, \rset, \vv{\v_\sset}, \vv{\v_\rset}, \anset)$ oracle then $\badv$ runs $\txn \gets \trans(\sk,\sset, \rset, \vv{\v_\sset}, \vv{\v_\rset}, \anset)$ with the difference that $\badv$ replace the proof with the output of $\pcalgostyle{O_{zk}(\txn[\inputs], \txn[\outputs])}$
        \item When $\adv$ queries the $\createAcctOracle(\userinfo, \anset)$ oracle then $\badv$ runs $\txn \gets \createAcct(\userinfo, \anset)$ with the difference that $\badv$ replace the proof with the output of $\pcalgostyle{O_{zk}(\txn[\inputs], \txn[\outputs])}$
        \item $\badv$ runs $b \gets \adv(state);$
    \end{enumerate}
    If $\adv$ answers Hybrid 0 then $\pcalgostyle{O_{zk}(\cdot)}$ is a prover oracle. If $\adv$ answers Hybrid 1 then $\pcalgostyle{O_{zk}(\cdot)}$ is a simulator oracle. So $\badv$ wins with probability $\epsilon$.
\end{proof}

\begin{lemma}
    Hybrid 2 and Hybrid 3 are indistinguishable.
\end{lemma}
\begin{corollary}
    Hybrid 9 and Hybrid 10 are indistinguishable.
\end{corollary}

\begin{proof}
    Note that $\adv$ cannot distinguish Hybrid 2 and Hybrid 3 from the fact that commitments are under different public key on the grounds that this breaks the key-anonymous property of the commitment scheme.
    Let $\adv$ be an adversary that can distinguish Hybrid 2 and Hybrid 3 with advantage $\epsilon$. We construct an adversary $\badv$ that breaks the indistinguishability property of the UPK scheme with probability $\epsilon$.

    In order to create $\badv$, we define five sub-hybrids. Let $h_0$ be Hybrid 2 and for each $i \in \{1,2,3,4\}$ $h_i$ would be a sub-hybrid where we replace the account $\acct_0, \acct_1, \acct_0^\prime, \acct_1^\prime$ respectively. In hybrid $h_4$ all of the accounts will be changed, therefore $h_4$ is Hybrid 3. Lets $\adv$ be an adversary that can distinguish  $h_i$ from $h_{i+1}$. Let $\acct_c$ be the account that we are replacing in this hybrid. Then: \\
    $\badv$ gets as input the tuple $(\acct^*, \acct_b)$ from the indistinguishability game and runs as follows:
    \begin{enumerate}
        \item $\badv$ generates $\state \gets \setup(\secpar)$.
        \item when $\adv$ uses the $\regOracle$ Oracle to create the initial account that share the same secret key with $\acct_c$, $\badv$ replaces this account with $\acct^*$.
        \item when $\adv$ uses $\transOracle$  or $\createAcctOracle$ Oracle to create the account $\acct_c$, $\badv$ replaces $\acct_c$ with $\acct_b$.
        \item $\badv$ reply to all other queries in the oracles as in the Hybrid $h_0$.
        \item $\badv$ outputs $b^\prime \gets \adv(\state)$.  
    \end{enumerate}
    We know that $\adv$ did not query the corrupt oracle on $\acct_c$ or on any other account that shares the same secret key with $\acct_c$ cause it would have immediately lost the anonymity game. Note that if $b=0$ then the distribution of the game is the same as hybrid $h_i$ and if $b = 1$ then the game has the same distribution as hybrid $h_{i+1}$. Hence $\badv$ answer $b^\prime$ and solves the indistinguishability game with probability $\epsilon$.
\end{proof}

\begin{lemma}
    Hybrid 3 and Hybrid 4 are indistinguishable.
\end{lemma}
\begin{corollary}
    Hybrid 10 and Hybrid 11 are indistinguishable.
\end{corollary}

\begin{proof}
    The only difference from this two Hybrids are the randomness to the commitments of the real participants accounts. Therefore, they produce a computationally indistinguishable distribution, due to the hiding property if the used commitment scheme.
\end{proof}
\begin{corollary}
    Hybrid 4 and Hybrid 5 are indistinguishable.\\
    Hybrid 11 and Hybrid 12 are indistinguishable.\\
    It can be proven the same way as Hybrid 2 and Hybrid 3 are indistinguishable.
\end{corollary}
\begin{corollary}
    Hybrid 5 and Hybrid 6 are indistinguishable.\\
    Hybrid 12 and Hybrid 13 are indistinguishable.\\
    It can be proven the same way as Hybrid 3 and Hybrid 4 are indistinguishable.
\end{corollary}
% \begin{lemma}
%     Hybrid 4 and Hybrid 5 are indistinguishable.
% \end{lemma}

% \begin{proof}
    
% \end{proof}

% \begin{lemma}
%     Hybrid 5 and Hybrid 6 are indistinguishable.
% \end{lemma}

% \begin{proof}
    
% \end{proof}

\begin{lemma}
    Hybrid 6 and Hybrid 13 are indistinguishable.
\end{lemma}


\begin{proof}
    Hybrid 6 and Hybrid 13 differ to (1) the accounts that are included in $\pset$ and in $\anset$ as well as to (2) the balances that are stored in the real participants' accounts in the challenge query ($\acct_i = \in \{\acct_0, \acct_1, \acct_0^\prime, \acct_1^\prime\}$). Concerning the former (1), in both Hybrids the $\inputs$ that $\adv$ sees is obtained by permuting $(\pset_x \cup \anset_x)$ with a random permutation $\psi$. But the union of these set in both cases ($x=\{0,1\}$) produces identical distributions. As a result $\adv$ cannot distinguish the two Hybrids from (1). The second change (2) produces a computationally indistinguishable distribution, due to the hiding property of the commitment scheme. Therefore, if $\adv$ could distinguish these Hybrids based on (2) then $\adv$ could break the hiding property of $\commit$.
\end{proof}

Using the above lemmas and the triangle inequality, we prove that there is not a PPT adversary $\adv$ that can distinguish Hybrid 0 and Hybrid 7 with more than negligible advantage.

\end{proof}

\subsubsection{Proof of theft prevention}
%The complete proof of theft prevention is provided in the Appendix; however, we'll outline the main intuition here. 
Intuitively, we argue that any PPT adversary $\adv$ capable of winning the theft-prevention game can be used to break either the unforgeability property of UPK scheme, the binding property of commitment scheme, or the soundness property of the NIZK proofs.

In order to win the theft-prevention game, $\adv$ should submit a transaction $\txn$ that either increases the total balance of the corrupted users, decreases the balance of honest users, or does not maintain preservation of value.
This can happen in the following ways:
The first way is if the adversary is able to transfer some amount from a honest user's account. However, this means that $\adv$ can compute the $\sk$ of the honest account, thus the unforgeability property of the UPK scheme is violated. 
Secondly,  if $\adv$ manages to transfer more coins than the corrupted account holds. But in order for such a transaction to be valid, the adversary should either be able to make a zk-proof that violates the soundness property, or to compute an opening to a commitment with balance different from the real one, hence breaking the binding property of the commitment scheme. 
The third way is by creating a transaction that breaks preservation of value, but in order for such a transaction to be valid, $\adv$ should again be able to construct an unsound zk-proof or break the binding property of the commitment scheme.

% \section{Full proof of theft prevention}

% //TODO: Add reference to definition.
\begin{theorem}
    AQQUA satisfies theft prevention, as defined in \autoref{def:theft-prevention}.
\end{theorem}
\begin{proof}
    Assume that there exists a PPT $\adv$ that wins the theft prevention game of Game~\ref{alg:theftprevention} with non-negligible probability. Using the notation of the game, we have that $\adv$ outputted a valid transaction $\txn$ that verifies and that results in one of the three winning conditions of the game being satisfied.
%     \begin{itemize}
%         \item[1.] $s^\prime_h < s_h$
%         \item[2.] $s^\prime_c > s_c$
%         \item[3.] $s^\prime_c + s^\prime_h > s_c + s_h$. 
%     \end{itemize}
    
%    We will focus on the first condition, since the other two follow similarly.

   We have that $\txn = (\inputs, \outputs, \pi)$, where $\pi$ is a ZK-proof for the relation $R(x,w)$ as defined in \autoref{fig:trans}, with $x = (\inputs, \outputs)$ and $w= (\sk, \varbl, \varout, \varin, \vv{\vbl}, \vv{\vout}, \vv{\vin}, \vv{r}, \psi, \isset^*, \irset^*, \ianset^*)$.
   
    From the soundness property of the NIZK argument of the $\trans$ algorithm, we can extract a witness $w^*= (\sk^*, \varbl^*, \cdots, \vv{\vbl^{*\prime}}, \cdots, \vv{r^*}, \cdots)$ such that $R(x, w^*)=1$.

    % //TODO: make the below more formal-need to study the defs
    % Furthermore, we note that after the execution of $\state^\prime \gets \applytxn(\txn, \state)$, we have that $\state^\prime$ is exactly like $\state$, but the accounts of $\inputs$ have been replaced with those of $\outputs$. 

    Let $\acct\in \inputs$ be the account such that $\verkp(\sk^*, \acct.\pk) = 1$.
    We divide into two cases.
   \begin{itemize}
    \item[1.] It holds that $\sk^* \in \honest$. In this case, we construct an adversary $\badv$ that breaks the unforgeability property of the UPK scheme with non-negligible probability.
    
    The reduction works as follows. The adversary $\badv$ takes as input a public key $\pk^*$. 
    It also keeps a directed tree with root $(\pk^*, 1)$ and whose nodes will be tuples of the form $(\pk, r)$. The tree will be updated so that for every edge of the form $((\pk_1, \cdot), (\pk_2, r_2))$ it will hold that $\verupd(\pk_2, \pk_1, r_2)=1$. 
    
    $\badv$ answers to $\adv$'s oracle queries as follows.
    \begin{itemize}
        \item When $\adv$ queries the $\regOracle$ oracle and this query results in the $\reg$ algorithm to generate $\sk^*$, $\badv$ replaces $\userinfo.\inpk$ with $\pk^*$, and when $\newacc$ is called in the procedure, $\badv$ gives as input $\pk^*$. The adversary $\badv$ stores the public key of the newly created account and the randomness used as a child of $(\pk^*, 1)$ in the tree. For the rest of the $\regOracle$ queries, $\badv$ answers honestly.
        \item When $\adv$ queries the $\createAcctOracle$ oracle for an account whose public key $\pk$ is contained in a leaf of the tree, $\badv$ answers honestly and adds a child to the leaf, composed of the updated public key of the updated account and the randomness used.
        \item When $\adv$ queries the $\transOracle$ oracle, the adversary $\badv$ acts as follows.
        \begin{itemize}
            % \item If the public keys of the accounts in $\sset$ are contained in leaves of the tree, $\badv$ replaces the zero-knowledge proof in $\txn$ with a simulated one.
            \item If the public keys of the accounts in $\sset$ are contained in leaves of the tree, $\badv$ creates an outputs set and creates a simulated proof for the transaction. $\badv$ also updates the tree by creating new children containing the updates of the public keys and the randomness. 
            \item If there exist public keys of accounts in the anonymity set that are contained in leaves of the tree, $\badv$ creates new children containing the updates of the public keys and the randomness.
        \end{itemize}
        \item When $\adv$ queries the $\applytxnOracle$ with a transaction whose inputs contain a leaf of the tree, $\badv$ uses the proof contained in the transaction to extract the witness. Then, $\badv$ creates new children for the updates of the public keys, storing also the randomness of the witness.
        \item For the rest of the oracle queries, $\badv$ answers honestly.
    \end{itemize}
    % Finally, when $\adv$ outputs the transaction $\txn$ of the theft prevention game, $\badv$ uses the extractor to extract a witness. The extracted witness contains a secret key $\sk^*$. The adversary $\badv$ finds the $\acct\in \inputs$ for which $\verkp(\sk^*, \acct.\pk)=1$, and finds the leaf $(\pk, r)$ of the tree for which $\acct.\pk = \pk$. Let $r^\prime$ be the multiplication of all randomnesses stored in the path from that leaf to the root. $\badv$ returns $(\pk, r^\prime)$.
    Finally, when $\adv$ outputs the transaction $\txn$ of the theft prevention game, $\badv$ finds the $\acct\in \inputs$ for which $\verkp(\sk^*, \acct.\pk)=1$, and finds the leaf $(\pk, r)$ of the tree for which $\acct.\pk = \pk$. Let $r^\prime$ be the multiplication of all randomnesses stored in the path from that leaf to the root. $\badv$ returns $(\pk, r^\prime)$.
    

    If $\adv$ wins the theft prevention game, we have that $\verkp(\pk, \sk^*) = 1$ and $\verupd(\pk, \pk^*, r^\prime)=1$. Since $\adv$ can win with non-negligible probability, $\badv$ breaks unforgeability with non-negligible probability.


    \item[2.] It holds that $\sk^*\in \corrupt$. 
    
    Assume w.l.o.g. that the transaction $\txn$ that $\adv$ outputs is the first transaction that results in winning the game (that is, there is no transaction submitted to $\applytxnOracle$ oracle prior to this point that would result in $\adv$ winning).

    Since $\adv$ wins the game, we have that the sum of the openings of the committed balances of all the accounts (stored in the bookkeeping) of $\inputs$ is different from those of $\outputs$. 
    
    % Let $\varbl_i, i = 1,\dots, n$ be the opening of the committed balance of the accounts in $\inputs$ in the bookkeeping, and similarly $\varbl^\prime_i$ for the accounts in $\outputs$.

    % From the soundness property of the NIZK argument of the $\trans$ algorithm, we have that for every $\acct\in \inputs$ and for the corresponding (w.r.t $\psi^*$ of $w^*$) account $\acct^\prime \in \outputs$ we have that $\verupdacc(\acct^\prime, \acct, \vbl^{\prime *}, \cdot, \cdot; \vv{r})=1 $. Furthermore, for the sender accounts of $\outputs$, we have that $\veract(\acct^\prime, \sk^*, \varbl^* + \vbl^{\prime *}, \cdot, \cdot)=1$.

    From the soundness property of the NIZK argument of the $\trans$ algorithm, we have that for every sender account $\acct^\prime$ of $\outputs$, $\veract(\acct^\prime, \sk^*, \varbl^* + \vbl^{\prime *}, \cdot, \cdot)=1$.

    Since $\veract$ returns $1$, and also $\sum_{\vbl^{\prime *} \in \vv{\vbl^{\prime *}}} \vbl^{\prime *} = 0$, and since $\adv$ wins the game, there exists an account $\acct \in \outputs$ for which $\acct.\com_{\varbl}$ has two different openings: one resulting from the bookkeeping, and one derived from the extracted witness (one of the values of the form $\varbl^* + \vbl^{\prime *}$ for some sender account). This trivially breaks the binding property of the commitment scheme.
    % there should exist an account $\acct \in \inputs \cup \outputs$ such that the opening 
    
    % In this case, the condition $s^\prime_h < s_h$ is equivalent to the existence of accounts $\acct_h \in \inputs$, $\acct_h^\prime \in \outputs$, for which there exists an $\sk_h \in \honest$ such that $\verkp(\sk_h, \acct_h.\pk) = 1$ and $\verkp(\sk_h, \acct_h.\pk^\prime) = 1$ and $\opencom_{\sk_h}(\acct.\com_{\varbl})>\opencom_{\sk_h}(\acct^\prime.\com_{\varbl})$. 
    
    % We construct an adversary $\badv$ that is able to break the binding property of the commitment scheme.
   \end{itemize}

\end{proof}

\subsubsection{Proof of audit soundness}
% The audit correctness property follows directly from the soundness and zero-knowledge properties of NIZK proofs.
Intuitively, we argue that any PPT adversary $\adv$ capable of winning the audit soundness game can be used to break either the binding property of commitment scheme or the soundness property of the NIZK proofs. 

In order to win the the audit soundness game, $\adv$ should either create a valid zero-knowledge proof without knowing the corresponding witness, or hide some of their accounts from the $\AA$. However, the former attack violates the soundness property of the zero-knowledge proof. The latter requires the $\adv$ to be able to open their commitment $\comm{\numaccs}$ to a different value, but this breaks again the binding property of the commitment scheme.

% \section{Full proof of audit soundness}

\begin{theorem}
    AQQUA satisfies audit soundness, as of \autoref{def:audit-soundness}
\end{theorem}
\begin{proof}
    Assume that there exist a PPT $\adv$ that wins the audit soundness game of Game \ref{alg:audit-soundness-game} with non-negligible probability. Using the notation of the game, we have that $\adv$ outputted a proof $\pi = (\pi_1, \pi_2)$ that verifies but $\adv$ is not compliant with the specified policy.
    
    % $\adv$ is given a policy $f$ with its auxiliary parameters $\aux$, an initial public key $\inpk$ and two snapshots from the blockchain $\snap_{1}, \snap_{2}$. Then $\adv$ constructs $\pi$ which is a ZK-proof for the relation $R(x,w)$ as defined in \autoref{fig:audit}, with $x = (\inpk, \snap_{1}, \snap_{2}, \{ \numaccs_j, {\comm{\numaccs_{j}}}, \{ {\acct_j}_i \}_{i=1}^{\numaccs_j}\}_{j=1}^{2}, \comm{\v}, \aux)$ and $w=(\sk, \v)$, where $\v, \aux$ are values that depend on the policy.
    % We have that $\pi$ is a ZK-proof for the relation $R(x,w)$ as defined in \autoref{fig:audit}, with $x = (\inpk, \snap_{1}, \snap_{2}, \{ \numaccs_j, {\comm{\numaccs_{j}}}, \{ {\acct_j}_i \}_{i=1}^{\numaccs_j}\}_{j=1}^{2}, \comm{\v}, \aux)$ and $w=(\sk, \v)$ ($\v, \aux$ depend on the policy).

    $\adv$ choose a policy $f$ with its auxiliary parameters $\aux$, an initial public key $\inpk$ and two snapshots from the blockchain $\snap_{1}, \snap_{2}$. Then $\adv$ constructs $\pi = (\pi_1, \pi_2)$ which as defined in \autoref{fig:audit} is a ZK-proof for the relations $R_1(x,w)$, with $x=(\inpk, \{ \numaccs_j, {\comm{\numaccs_{j}}}, \{ \acct_{ji} \}_{i=1}^{\numaccs_j}\}_{j=1}^{2})$ and $w=(\sk)$ and $R_2(x,w)$, with $x=(\{ \acct_{1i} \}_{i=1}^{\numaccs_j}, \{ \acct_{2i} \}_{i=1}^{\numaccs_j}, \comm{\v}, \aux)$ and $w=(\sk, \v)$, where $\v, \aux$ are values that depend on the policy.

    From the soundness property of the NIZK argument of the $\pi_1$, we can extract a witness $w^* = \sk^*$ such that $R_1(x, w^*)$ = 1. We have that every $\pk \in \{\inpk\} \cup \{\acct_{ji}.\pk\}_{i=1}^{\numaccs_j}$,  $\verkp(\sk^*, \pk)$. Therefore similarly to theft-prevention proof we can prove that if $\sk^* \in \honest$ then $\adv$ can be used to break the unforgeability property of UPK scheme. Else if $\sk^* \in \corrupt$ then since $\adv$ wins the game, we have that the opening to the commitment of $\comm{\numaccs}$ is different from the one that resulting from bookkeeping. This trivially breaks the binding property of the commitment scheme.

    From the soundness property of the NIZK argument of the $\pi_2$, we can extract a witness $w^* = \v^*$ such that $R_1(x, w^*)$ = 1. Again since $\adv$ wince the game the sum of the openings of the commited value of all the accounts that belongs to $\adv$ is different from the one that resulting from bookkeeping, so this breaks the binding property of the commitment scheme.
    % From the soundness property of the NIZK argument of the $\audit$ algorithm, we can extract a witness $w^* = (\sk^*, \v^*)$ such that $R(x, w^*)$ = 1. 

    % We have that every $\pk \in \{\inpk\} \cup \{\acct_{j_i}.\pk\}_{i=1}^{\numaccs_j}$,  $\verkp(\sk^*, \pk)$. We divide into two cases based on wether $\sk^* \in \honest$ or $\sk^* \in \corrupt$.

    % Similarly to theft-prevention proof:
    % \begin{itemize}
    %     \item if $\sk^* \in \honest$ we can prove that $\adv$ can be used in order to break the unforgeability property of UPK scheme.
    %     \item if $\sk^* \in \corrupt$ we can prove that $\adv$ trivially breaks the binding property of one of the commitments $\comm{\numaccs}$ or $\comm{\v}$.
    % \end{itemize}
    
\end{proof}
