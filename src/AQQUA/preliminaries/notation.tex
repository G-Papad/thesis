% //TODO I think that notation should apply to basic stuff and not system related.
\subsection{Notation}
% TODO: add explanation for 'acct.pk' (when we have a tuple, a=(b, c, ...) we denote a.b = b)
We denote by $\secpar$ the security parameter. 
We denote by $\msgspace$ the message space and by $\randspace$ the randomness space of our cryptographic schemes.
$\vset = \{0,\dots,V\}$ is the set that defines the range of valid currency values, where $V$ is an upper bound on the maximum possible number of coins in the system ($|\vset| \ll |\msgspace|$). 
When an element $x$ is sampled uniformly at random from a set $\mathcal{X}$, we write $x \sample \mathcal{X}$.
Given a tuple $t = (a,b)$ we refer to its parts using the dot notation, i.e. $t.a$ or $t.b$.
We  denote $(a^x,b^x)$ as $t^x = (a,b)^x$.
% We denote the initial public key with which the user has initially registered in the system as $\inpk$, the current public key of a user's account $\acct$ as $\pk$ and the output public key of update algorithms as $\pk'$.
% Also, the secret key of the user as $\sk$. 
% We use $n$ to denote the number of users that belong to an anonymity set $\anset$ and the participants of the transactions as $\pset$ 
% //TODO is n fixed?
% The $\numaccs$ states the number of accounts are related to a specific user.
% A commitment to a value $v$ is denoted by $\comm{v}$. 


