\subsubsection{Delete Account Algorithm}
Allowing users to delete zero-balance accounts reduces the storage overhead of AQQUA, since accounts that have no balance left to spend might be abandoned and thus not needed to be stored in the $\utxoset$. Furthermore, due to the fact that senders usually create new accounts for their intended recipients, the number of accounts in the $\utxoset$ increases if the option to remove zero-balance accounts is not given. We note that users should be incentivized to delete the zero-balance accounts they own and don't need to keep. The mechanism to do so is left for future work.

In order to delete an account, the information containing the total amount $\varout, \varin$ sent and received by the account must be transferred to another account $\accttotransfer$ of the corresponding owner. In order to hide $\accttotransfer$ an anonymity set is included. % //TODO: add/explain why this is needed.

The detailed description of the $\deleteAcct$ algorithm is depicted in \autoref{fig:deleteAcct}. The algorithm takes as input the secret key $\sk$, the account to be deleted $\accttodelete$, the account $\accttotransfer$ to which $\varout, \varin$ of $\accttodelete$ will be transferred, and anonymity sets $\anset_1$ for the $\utxoset$ and $\anset_2$ for the $\usersset$ respectively. It returns a transaction $\txnda = (\inputs, \outputs, \pi)$.

\begin{boxfig}{\label{fig:deleteAcct}{The $\deleteAcct$ algorithm.}}
/ 
The algorithm $\deleteAcct(\sk, \userinfo, \accttodelete, \accttotransfer, \anset_1, \anset_2)$ performs the following steps:
\begin{enumerate}
    % \item For the account $\accttodelete$, calculate the opening of the commitments $\com_{\varbl}, \com_{\varout}, \com_{\varin}$, denoted $\varbl_d, \outdelete, \indelete$, using the secret key $\sk$.
    \item For the account $\accttodelete$, calculate the opening of the commitments $\accttodelete.\com_{\varout}, \accttodelete.\com_{\varin}$, denoted $\outdelete, \indelete$, using the secret key $\sk$.
    \item Let  $\inputs_\utxoset = \{\accttotransfer\} \cup \anset_1$ in some canonical order.
    Let $c^*, {{\ianset}_1}$ denote the indices of the account to be added the information and the accounts of the anonymity set in this list.
    \item Construct $\vv{\vbl}, \vv{\vout}, \vv{\vin}$ as follows:
    \begin{itemize}
        \item $\vv{\vbl} = 0 \ \forall i \in \{c^*\}\cup {\ianset}_1$ 
        \item $\vv{\vout} = 0 \ \forall i \in {{\ianset}_1} \text{ and } {\vout}_{c^*} = \outdelete$
        \item $\vv{\vin} = 0 \ \forall i \in {{\ianset}_1} \text{ and } {\vin}_{c^*} = \indelete$
    \end{itemize}
    \item Pick $r_1, r_2, r_3, r_4 \sample \randspace$. and let $\vv{r} = (r_1, r_2, r_3, r_4)$. Let $\outputs_\utxoset$ be the output of $\updacc(\inputs_\utxoset,\vv{\vbl}, \vv{\vout}, \vv{\vin}; \vv{r})$ in some canonical order.
    \item Let $\psi:[\num]\rightarrow [\num]$ be the implicit permutation mapping $\inputs_\utxoset$ into $\outputs_\utxoset$; such that accounts $\acct_i \in \inputs_\utxoset$ and $\acct^\prime_{\psi(i)}\in \outputs_\utxoset$ share the same secret key.
    
    %//TODO: fic alignment
    \item Form a zero-knowledge proof $\pi_1$ of the relation $R(x, w)$, where \\
     $x = (\accttodelete, \inputs_\utxoset, \outputs_\utxoset), w = (\sk, \outdelete, \indelete, \vv{r}, \psi, c^*, {{\ianset}_1})$, and $R(x, w) = 1$ if 
    % for all $i \in [\num], {\acct}_i \in \inputs_\utxoset, {\acct'}_{\psi(i)} \in \outputs_\utxoset$ we have that:
    {\begin{align*}
        & \verkp(\sk, \accttodelete.\pk) = 1 \land \verkp(\sk, \acct_{c^*}.\pk) = 1\\
        & \land \verupdacc({\acct'}_{\psi(i)},{\acct}_i,0,0,0;\vv{r})=1 \ \forall i \in {{\ianset}_1} \\
        & \land \verupdacc({\acct'}_{\psi(c^*)},{\acct}_{c^*}, 0, \outdelete, \indelete ;\vv{r})=1 \\
        & \land \vercom(\accttodelete.\pk, \accttodelete.\com_{\varbl}, (\sk, 0)) = 1 \\
    \end{align*}}

    \item Let $\inputs_\usersset = \{\userinfo\} \cup \anset_2$ in some canonical order. Let $s^*, {{\ianset}_2}$ denote the indices of the chosen initial public key for which we wish to construct the new account, and the anonymity set in this list. 
    \item Construct $\vv{\v}$ as follows: $\v_i = 0 \ \forall i \in {{\ianset}_2}$ and $\v_{s^*} = -1$. 
    \item Pick $r \sample \randspace$ and let $\outputs_\usersset$ be the output of $\upduser(\inputs_\usersset, \vv{\v};r)$.
    \item Form a zero-knowledge proof $\pi_2$ of the relation $R(x, w)$, where 
     $x=(\inputs_\usersset, \outputs_\usersset), w = (\sk, r, s^*, {{\ianset}_2})$ and $R(x,w) = 1$ if $\forall i \in \{s^*\} \cup {{\ianset}_2} \ \userinfo_i \in \inputs_\usersset, \userinfo'_i \in \outputs_\usersset$ we have that:
        {\begin{align*}
            & \verkp(\sk, \userinfo_{s^*}.\inpk) = 1 \\
            & \land \verupduser(\userinfo'_i, \userinfo_i, 0;r) = 1 \ \forall i \in {{\ianset}_2} \\
            & \land \verupduser(\userinfo'_{s^*}, \userinfo_{s^*}, -1;r) = 1 \\
        \end{align*}}

\end{enumerate}


The final transaction returned by the algorithm is \\ $\txnda = (\inputs_\utxoset, \outputs_\utxoset, \inputs_\usersset, \outputs_\usersset, \pi = (\pi_1, \pi_2))$.

\end{boxfig}