\section{Overview}

\subsubsection{Objective}
The solutions proposed in the literature that aim to combine privacy and auditability using a general auditor rather than a centralized authority either suffer from limited scalability or do not provide full privacy.

Therefore, we aim to construct an efficient, anonymous, confidential and auditable system that can also support concurrent transactions and maintain a stable state size regardless of the number of users or transaction history. To create such a system, we propose AQQUA, which extends the Quisquis \cite{fauzi2019quisquis} DPS system with a general auditor. Thus, AQQUA combines the anonymity of Quisquis with the policy expressiveness and regulation of PGC \cite{PGC}. This means that the auditor can perform queries on the upper limit of the amount sent/received by the user in a given period, on the non-participation of a user in a given transaction or period, as well as on the exact value sent/received in a transaction.

We chose to extend Quisquis because, as already mentioned, it is a fully private system, offering both anonymity and confidentiality, while having a constant storage cost with respect to the number of transactions. Moreover, due to the fact that the anonymity set used to hide the participants of a transaction does not include all the users of the system, as in zkLedger, Quisquis can support concurrent transactions. 

\subsubsection{Challenges}
The design of AQQUA had to overcome the following key challenge to enable auditing functionalities while still preserving user privacy. Due to the anonymity property, users can hide their accounts and consequently the amounts necessary for the auditing process. In addition, since Quisquis is permissionless and private, an authority cannot enforce some effective penalties for non-compliant users. 

To overcome this challenge, we introduce a registration functionality in Quisquis through a registration authority. This means that users must first register with the systems and provide their real-world credentials. They can then create new, unlinked accounts that are used to transact within the system. The registration functionality provides a way for the system to know which users are using the system and to penalize them in a way that is outside the scope of the system. In addition, AQQUA splits the state into two sets. It keeps the UTXO set used in Quisquis, which contains the user's unspent accounts, but also adds a new set that contains the user's public registration information along with the necessary information to ensure that users cannot hide information during the audit process.