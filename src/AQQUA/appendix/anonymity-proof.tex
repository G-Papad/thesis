% \section{Full proof of anonymity}

% Before we give the proof of anonymity, we first recall a  definition for indistinguishability of UPK scheme \cite{fauzi2019quisquis}.
% \begin{definition}\label{def:UPKindistinguishability} 
%     The \emph{advantage} of the adversary in winning the indistinguishability game is defined as:
%     $
%         \advantage{ind}{\adv} = \mid \Pr[\exper^{ind}_{\adv}((\secpar))=1] - \dfrac{1}{2} \mid
%     $

%     A DPS satisfies \emph{indistinguishability} if for every PPT adversary $\adv$, $\advantage{ind}{\adv}$ is negligible in  $\secpar$.
% \end{definition}
% \begin{game}[htbp]
    % \DontPrintSemicolon
    % \SetAlgoLined
%     \caption{Indistinguishability game $\exper^{ind}_{\adv}(\secpar)$}
%     \label{alg:UPKindistinguishability-game}
    % \SetKwInOut{Input}{Input}
    % \SetKwInOut{Output}{Output}
%     \Input{$\secpar$}
%     \Output{$\{0,1\}$}
%     \BlankLine
%     $b\gets \{0, 1\}$ \;
%     $(\pk^*, \sk^*) \gets \kgen()$\;
%     $r \sample \randspace$\;
%     $\pk_0 \gets \upd(\pk^*;r)$\;
%     $(\pk_1, \sk_1) \gets \kgen()$\;
%     $b^\prime \gets \adv(\pk^*, \pk_b)$\;
%     \Return{$(b = b^\prime)$}
% \end{game}
% Note that in indistinguishability game the challenger can update many times the $\pk^*$ before creating $\pk_0$ due to the fact that even with more updates the $pk_0$ can be described as an update of $\pk^*$ with a different randomness.

% \begin{lemma}
%     The constructed UPK scheme satisfies \ref{def:UPKindistinguishability} if the DDH assumption holds in $(\group, g, p)$.
% \end{lemma}
% Proof of this lemma can be found on \cite{fauzi2019quisquis}.

\begin{theorem}
    AQQUA satisfies anonymity, as defined in \autoref{def:anonymity}
\end{theorem}
\begin{proof}
We prove the theorem using a sequence of 14 hybrid games, as follows. Hybrid 0 and Hybrid 7 are the anonymity game for $b=0, b=1$ respectively. Each of the rest hybrids differs in oracles' functionalities in a way that the successive hybrids are indistinguishable from the view of the adversary. We use these hybrids to prove that the adversary cannot distinguish anonymity game for $b=0$ and anonymity game with $b=1$. \\
\textbf{Hybrid 0.} The anonymity game for $b=0$. \\
% 
\textbf{Hybrid 1.} Same as Hybrid 0, but here we run the NIZK extractor on each transaction generated by the adversary. That means, when $\adv$ runs the $\applytxnOracle(\txn)$ Oracle, the Oracle verifies $\txn$ by running $\vertxn(\txn, \state)$ 
% or $\vercreateAcct(\txn, \state)$ 
%or $\verdeleteAcct(\txn, \state)$ 
depending on the transaction $\txn$ and if it is successful the oracle runs $\state^\prime \gets \applytxn(\state, \txn)$, as well as uses the NIZK extractor to extract the witness used to generate $\txn$, including $\sk$. \\
% 
\textbf{Hybrid 2.} Same as Hybrid 1, but here the zero-knowledge arguments of the each transaction is replaced with the output of the corresponding simulator of the zero-knowledge property of NIZK. In order to achieve this we change the following oracles' functionality:
\begin{itemize} 
    \item when $\adv$ or the challenger creates $\txn$ through the $\transOracle(\sset, \rset, \vv{\v_\sset}, \vv{\v_\rset}, \anset)$ Oracle, the Oracle runs $\txn \gets \trans(\sk,\sset, \rset, \vv{\v_\sset}, \vv{\v_\rset}, \anset)$, but replaces the zero-knowledge arguments in $\txn$ with a simulated argument.\\
    
    \item when $\adv$ or the challenger creates $\txn$ through the $\createAcctOracle(\userinfo, \anset)$ Oracle, the Oracle runs $\txn \gets \createAcct(\userinfo, \anset)$, but replaces the zero-knowledge arguments in $\txn$ with a simulated argument.\\
    % \item when $\adv$ or the challenger creates $\txn$ through the $\deleteAcctOracle(\userinfo, \acct_c, \acct_d, \anset)$ Oracle, the Oracle runs $\txn \gets \deleteAcct(\sk, \acct_d, \acct_c, \anset)$, but replaces the zero-knowledge arguments in $\txn$ with a simulated argument.\\
\end{itemize}
% 
\textbf{Hybrid 3.} Same as Hybrid 2, but now the challenger replaces the potential senders' and receivers' accounts of the challenge transaction $\txn_0$ $(\acct_0, \acct_1, \acct_0^\prime, \acct_1^\prime)$, with new accounts that have a freshly created key pair ($\sk, \pk)$ derived from the output of the $\kgen()$. 
In order to achieve this we change the following oracles' functionality:
\begin{itemize} 
    \item when $\adv$ creates one of these accounts $\acct_i$ through the $\transOracle$ Oracle (these accounts are presented in $\txn.\outputs$), the Oracle runs $\txn \gets \trans(\sk,\sset, \rset, \vv{\v_\sset},$ \\ $ \vv{\v_\rset}, \anset), (\pk_i^\prime, \sk_i^\prime) \gets \kgen$ and then return $\txn^\prime$, where $\txn^\prime = \txn$ except that each $\acct_i \in \{\acct_0, \acct_1, \acct_0^\prime, \acct_1^\prime\}$ is replaced with $\acct_i^\prime = (\pk_i^\prime, \com_{\varbl i}, \com_{\varout i},$ \\ $ \com_{\varin i})$.\\
    
    \item when $\adv$ creates one of these accounts $\acct_i$ through the $\createAcctOracle$ Oracle, the Oracle runs $\txn \gets \createAcct(\userinfo, \anset), (\pk_i^\prime, \sk_i^\prime) \gets \kgen$ and then return $\txn^\prime$, where $\txn^\prime = \txn$ except that each $\acct_i \in \{\acct_0, \acct_1, \acct_0^\prime, \acct_1^\prime\}$ is replaced with 
    % $\acct_i^\prime = (\pk_i^\prime, \com_{\varbl i}, \com_{\varout i}, \com_{\varin i})$.\\
    $\acct_i^\prime = (\pk_i^\prime, \comm{0}, \comm{0}, \comm{0})$.\\
\end{itemize}
% \textbf{Hybrid 2 | 6.} Same as Hybrid 1 | Hybrid 5, but now the challenger replace the real participants of $\txn_0$, meaning the accounts $(\acct_0, \acct_1, \acct_0^\prime, \acct_1^\prime)$, with fresh accounts that have a newly created public key derived from the output of the $\kgen$ and commitments of the same value under the new $\pk$ and a different randomness. That is, when $\adv$ creates one of these accounts $\acct_i$ through the $\transOracle$ Oracle (when these accounts are presented in the outputs of a $\txn$), the Oracle runs $\txn \gets \trans(\sk,\sset, \rset, \vv{\v_\sset}, \vv{\v_\rset}, \anset), (\pk^\prime, \sk^\prime) \gets \kgen, \com_\varbl \gets \commit(\pk^\prime,\acct_i.\varbl), \com_\varout \gets \commit(\pk^\prime,\acct_i.\varout), \com_\varin \gets \commit(\pk^\prime,\acct_i.\varin)$ and then return $\txn^\prime$, where $\txn^\prime = \txn$ except that each $\acct_i \in \{\acct_0, \acct_1, \acct_0^\prime, \acct_1^\prime\}$ is replaced with $\acct^\prime = (\pk^\prime, \com_\varbl, \com_\varout, \com_\varin)$.\\
% 
\textbf{Hybrid 4.} Same as Hybrid 3, but here the challenger replaces also the commitments of the accounts $(\acct_0, \acct_1, \acct_0^\prime, \acct_1^\prime)$ with newly created commitments to the same values with different randomness. 
In order to achieve this we change the following oracles' functionality:
\begin{itemize} 
    \item when $\adv$ creates one of these accounts $\acct_i$ through the $\transOracle$ Oracle (these accounts are presented in $\txn.\outputs$), the Oracle runs $\txn \gets \trans(\sk,\sset, \rset, \vv{\v_\sset},$ \\ $ \vv{\v_\rset}, \anset), (r_1, r_2, r_3) \sample \randspace,
    \varbl_i \gets \opencom(\sk, \acct_i.\com_{\varbl}), \varout_i \gets \opencom(\sk, $ \\ $\acct_i.\com_{\varout}), \varin_i \gets \opencom(\sk, \acct_i.\com_{\varin}), 
    \com_\varbl^\prime \gets \commit(\pk^\prime, \varbl_i; r_1), $ \\ $\com_\varout^\prime \gets \commit(\pk^\prime,\varout_i;r_2), \com_\varin^\prime \gets \commit(\pk^\prime,\varin_i; r_3)$ and then return $\txn^\prime$, where $\txn^\prime = \txn$ except that each $\acct_i \in \{\acct_0, \acct_1, \acct_0^\prime, \acct_1^\prime\}$ is replaced with $\acct^\prime = (\pk, \com_\varbl^\prime, \com_\varout^\prime, \com_\varin^\prime)$. ($\pk = \pk^\prime$ as in the Hybrid 3). \\
    
    \item when $\adv$ creates one of these accounts $\acct_i$ through the $\createAcctOracle$ Oracle, the Oracle runs $\txn \gets \createAcct(\userinfo, \anset), (r_1, r_2, r_3) \sample \randspace,  \com_\varbl^\prime \gets \commit(\pk^\prime0; r_1), \com_\varout^\prime \gets \commit(\pk^\prime,0;r_2), \com_\varin^\prime \gets \commit(\pk^\prime,0; r_3)$ and then return $\txn^\prime$, where $\txn^\prime = \txn$ except that each $\acct_i \in \{\acct_0, \acct_1, \acct_0^\prime, \acct_1^\prime\}$ is replaced with $\acct^\prime = (\pk, \com_\varbl^\prime, \com_\varout^\prime, \com_\varin^\prime)$. ($\pk = \pk^\prime$ as in the Hybrid 3). \\
\end{itemize}
% 
\textbf{Hybrid 5.} Same as Hybrid 4, but here also the updated accounts of $(\acct_0, \acct_1, \acct_0^\prime, \acct_1^\prime)$ in the challenge $\txn.\outputs$ are replaced by accounts with freshly created public key $\pk^\prime$.\\
% 
\textbf{Hybrid 6.} Same as Hybrid 5, but here also the updated accounts of $(\acct_0, \acct_1, \acct_0^\prime, \acct_1^\prime)$ in the challenge $\txn.\outputs$ are replaced by accounts with freshly created commitments to the same value.\\

Afterwards, we create Hybrids 7-13 that are the same with Hybrids 0-6 with the difference that are made for the anonymity game with $b=1$.

Note that in Hybrid 6 and in  Hybrid 13 all accounts of the potential senders' and receivers' accounts of the challenge transaction $\txn_b$ (both in $\inputs$ and $\outputs$) are fresh accounts, where in $\outputs$ have been generated with values corresponding to the case $b=0$ | $b=1$.

Now we will prove that $\adv$ has negligible advantage of distinguish Hybrid 0 and Hybrid 7. \\
% //TODO: Prove each Lemma:
% Lemma: Hybrid 1 and Hybrid 2 are indistinguishable -> zk property \\
% Lemma: Hybrid 2 and Hybrid 3 are indistinguishable -> UPK  indistinguishability + key-anonymous of commitments\\
% Lemma: Hybrid 3 and Hybrid 4 are indistinguishable -> hiding of commitments \\
% Lemma: Hybrid 4 and Hybrid 5 are indistinguishable -> UPK  indistinguishability + key-anonymous of commitments\\
% Lemma: Hybrid 5 and Hybrid 6 are indistinguishable -> hiding of commitments \\
% Lemma: Hybrid 6 and Hybrid 13 are indistinguishable -> hiding of commitments \\
\begin{lemma}
    Hybrid 0 and Hybrid 1 are indistinguishable.
\end{lemma}
\begin{corollary}
    Hybrid 7 and Hybrid 8 are indistinguishable.
\end{corollary}
\begin{proof}
The adversary's view in the two hybrids' game are identical.
\end{proof}

\begin{lemma}
    Hybrid 1 and Hybrid 2 are indistinguishable.
\end{lemma}
\begin{corollary}
    Hybrid 8 and Hybrid 9 are indistinguishable.
\end{corollary}
\begin{proof}
    Let $\adv$ be an adversary that can distinguish Hybrid 1 and Hybrid 2 with advantage $\epsilon$. We construct an adversary $\badv$ that breaks the zero-knowledge property of the NIZK proof $\pi$ of transaction $\txn$ with probability $\epsilon$.
    
    Let $\pcalgostyle{O_{zk}(\cdot)}$ be an oracle that on input $(\txn.\inputs, \txn.\outputs)$ creates a valid zero-knowledge proof for the transaction. Then $\badv$ wins if they can decide wether $\pcalgostyle{O_{zk}(\cdot)}$ is a prover or simulator oracle.
    
    $\badv$ takes as input the $\pcalgostyle{O_{zk}(\cdot)}$ and runs as follows:
    \begin{enumerate}
        \item $\badv$ generates $\state \gets \setup(\secpar)$;
        \item When $\adv$ queries the $\transOracle(\sset, \rset, \vv{\v_\sset}, \vv{\v_\rset}, \anset)$ oracle then $\badv$ runs $\txn \gets \trans(\sk,\sset, \rset, \vv{\v_\sset}, \vv{\v_\rset}, \anset)$ with the difference that $\badv$ replace the proof with the output of $\pcalgostyle{O_{zk}(\txn[\inputs], \txn[\outputs])}$
        \item When $\adv$ queries the $\createAcctOracle(\userinfo, \anset)$ oracle then $\badv$ runs $\txn \gets \createAcct(\userinfo, \anset)$ with the difference that $\badv$ replace the proof with the output of $\pcalgostyle{O_{zk}(\txn[\inputs], \txn[\outputs])}$
        \item $\badv$ runs $b \gets \adv(state);$
    \end{enumerate}
    If $\adv$ answers Hybrid 0 then $\pcalgostyle{O_{zk}(\cdot)}$ is a prover oracle. If $\adv$ answers Hybrid 1 then $\pcalgostyle{O_{zk}(\cdot)}$ is a simulator oracle. So $\badv$ wins with probability $\epsilon$.
\end{proof}

\begin{lemma}
    Hybrid 2 and Hybrid 3 are indistinguishable.
\end{lemma}
\begin{corollary}
    Hybrid 9 and Hybrid 10 are indistinguishable.
\end{corollary}

\begin{proof}
    Note that $\adv$ cannot distinguish Hybrid 2 and Hybrid 3 from the fact that commitments are under different public key on the grounds that this breaks the key-anonymous property of the commitment scheme.
    Let $\adv$ be an adversary that can distinguish Hybrid 2 and Hybrid 3 with advantage $\epsilon$. We construct an adversary $\badv$ that breaks the indistinguishability property of the UPK scheme with probability $\epsilon$.

    In order to create $\badv$, we define five sub-hybrids. Let $h_0$ be Hybrid 2 and for each $i \in \{1,2,3,4\}$ $h_i$ would be a sub-hybrid where we replace the account $\acct_0, \acct_1, \acct_0^\prime, \acct_1^\prime$ respectively. In hybrid $h_4$ all of the accounts will be changed, therefore $h_4$ is Hybrid 3. Lets $\adv$ be an adversary that can distinguish  $h_i$ from $h_{i+1}$. Let $\acct_c$ be the account that we are replacing in this hybrid. Then: \\
    $\badv$ gets as input the tuple $(\acct^*, \acct_b)$ from the indistinguishability game and runs as follows:
    \begin{enumerate}
        \item $\badv$ generates $\state \gets \setup(\secpar)$.
        \item when $\adv$ uses the $\regOracle$ Oracle to create the initial account that share the same secret key with $\acct_c$, $\badv$ replaces this account with $\acct^*$.
        \item when $\adv$ uses $\transOracle$  or $\createAcctOracle$ Oracle to create the account $\acct_c$, $\badv$ replaces $\acct_c$ with $\acct_b$.
        \item $\badv$ reply to all other queries in the oracles as in the Hybrid $h_0$.
        \item $\badv$ outputs $b^\prime \gets \adv(\state)$.  
    \end{enumerate}
    We know that $\adv$ did not query the corrupt oracle on $\acct_c$ or on any other account that shares the same secret key with $\acct_c$ cause it would have immediately lost the anonymity game. Note that if $b=0$ then the distribution of the game is the same as hybrid $h_i$ and if $b = 1$ then the game has the same distribution as hybrid $h_{i+1}$. Hence $\badv$ answer $b^\prime$ and solves the indistinguishability game with probability $\epsilon$.
\end{proof}

\begin{lemma}
    Hybrid 3 and Hybrid 4 are indistinguishable.
\end{lemma}
\begin{corollary}
    Hybrid 10 and Hybrid 11 are indistinguishable.
\end{corollary}

\begin{proof}
    The only difference from this two Hybrids are the randomness to the commitments of the real participants accounts. Therefore, they produce a computationally indistinguishable distribution, due to the hiding property if the used commitment scheme.
\end{proof}
\begin{corollary}
    Hybrid 4 and Hybrid 5 are indistinguishable.\\
    Hybrid 11 and Hybrid 12 are indistinguishable.\\
    It can be proven the same way as Hybrid 2 and Hybrid 3 are indistinguishable.
\end{corollary}
\begin{corollary}
    Hybrid 5 and Hybrid 6 are indistinguishable.\\
    Hybrid 12 and Hybrid 13 are indistinguishable.\\
    It can be proven the same way as Hybrid 3 and Hybrid 4 are indistinguishable.
\end{corollary}
% \begin{lemma}
%     Hybrid 4 and Hybrid 5 are indistinguishable.
% \end{lemma}

% \begin{proof}
    
% \end{proof}

% \begin{lemma}
%     Hybrid 5 and Hybrid 6 are indistinguishable.
% \end{lemma}

% \begin{proof}
    
% \end{proof}

\begin{lemma}
    Hybrid 6 and Hybrid 13 are indistinguishable.
\end{lemma}


\begin{proof}
    Hybrid 6 and Hybrid 13 differ to (1) the accounts that are included in $\pset$ and in $\anset$ as well as to (2) the balances that are stored in the real participants' accounts in the challenge query ($\acct_i = \in \{\acct_0, \acct_1, \acct_0^\prime, \acct_1^\prime\}$). Concerning the former (1), in both Hybrids the $\inputs$ that $\adv$ sees is obtained by permuting $(\pset_x \cup \anset_x)$ with a random permutation $\psi$. But the union of these set in both cases ($x=\{0,1\}$) produces identical distributions. As a result $\adv$ cannot distinguish the two Hybrids from (1). The second change (2) produces a computationally indistinguishable distribution, due to the hiding property of the commitment scheme. Therefore, if $\adv$ could distinguish these Hybrids based on (2) then $\adv$ could break the hiding property of $\commit$.
\end{proof}

Using the above lemmas and the triangle inequality, we prove that there is not a PPT adversary $\adv$ that can distinguish Hybrid 0 and Hybrid 7 with more than negligible advantage.

\end{proof}