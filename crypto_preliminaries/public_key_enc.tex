\subsection{Public Key Encryption}

The components of public key cryptography will be strictly defined in this section. 

A cryptosystem is a tuple of the following three algorithms $(\kgen, \enc, \dec)$ such that if $m$ a message then:
\begin{itemize}
    \item $k \gets \kgen()$
    \item $c \gets \enc(k,m)$
    \item $m \gets \dec(k,c)$
    \item Also it holds that $\dec(\enc(m)) = m$.
\end{itemize}

A public key encryption cryptosystem has the following properties:

The $\kgen$ algorithm produces a pair of keys, denoted as $(\pk, \sk)$. $\sk$ is the private key and must remain hidden, while $\pk$ is the public key and is known to everyone using the system. 
When someone wants to send a message $m$ to a particular recipient, the sender encrypts $m$ using the $\enc$ algorithm under the recipient's $\pk$. Only the owner of the corresponding $\sk$ can use the $\dec$ algorithm and decrypt the encrypted message to the original plaintext $m$.

The positive aspect of public key cryptography is that it does not require an exchange of keys between each user of the system; as soon as a new person joins the network, he only needs to publish his public key and can receive messages from any other user.

An example of public key encryption scheme is the ElGamal cryptosystem.

% \section{El Gamal Encryption}

\subsubsection{Standard ElGamal}

\subsubsection{Exponential ElGamal}

\subsubsection{Twisted El Gamal}
\label{TwistedElGamal}

The differences between ElGamal and twisted ElGamal encryption is that in the latter: (i) the message is encoded over another generator $h$ that the logarithmic relation with $g$ is unknown (ii) the key encapsulation and session key are switched. It is been proved in \cite{PGC} that is as secure and efficient as standard ElGamal as well as it's ciphertext can be used as direct input to bulletproofs.

More specifically twisted ElGamal consists of the following algorithms:
\begin{itemize}
    \item $\setup(1^\lambda)$: run $(\group, g, p) \gets GroupGen(1^\lambda)$, pick $h \sample \group^*$, set $pp = (\group, g,h,p)$.
    \item $\kgen(pp)$: choose $\sk \sample \ZZ_p$, set $\pk = g^\sk$. 
    \item $\enc(\pk,\msg;r)$: compute $X = \pk^r, Y=g^r h^m$, output $C=(X,Y)$.
    \item $\dec(\sk, C)$: parse $C=(X,Y)$, compute $h^m = Y/X^{sk^{-1}}$. Recover $m$ from $h^m$. 
\end{itemize}

\subsubsection{ElGamal}
ElGamal is based on DLOG assumptions. It consists of the following algorithms:
\begin{itemize}
    \item \textbf{Key-pair Generation} - $\kgen()$:
    \begin{enumerate}
        \item Choose $p,q$ large primes such that $q|(p-1)$ and a generator $g$ of subgroup $\group$ with order $q$ of group $\ZZ_p^*$.
        \item Choose random $x \in \ZZ_q$
        \item Calculate $y = g^x$ mod $p$
        \item Return key-pair $(\sk, \pk) = (x,y)$ 
    \end{enumerate}

    \item \textbf{Encryption} - $\enc(\pk, m)$:
    \begin{enumerate}
        \item Sample $r \sample \ZZ_q$
        \item Calculate $G = g^r$ mod $p$
        \item Calculate $M = my^r$ mod $p$
        \item Return ciphertext $c = (G,M)$
    \end{enumerate}

    \item \textbf{Decrypt} - $\dec(\sk, c)$:
    \begin{enumerate}
        \item Given ciphertext $c = (G,M)$ and the owner of $\sk = x$ can calculate $m = \frac{M}{G^x}$.
    \end{enumerate}
    It is obvious that $\dec(\sk, \enc(m)) = m$ since:
    \begin{equation*}
        \frac{M}{G^x} = \frac{my^r}{{(g^r)}^x} = \frac{{m(g^)}^r}{g^{xr}} = m
    \end{equation*}
\end{itemize}

There is also a variation of ElGamal encryption that returns as message $g^m$ instead of $m$ that is called exponential ElGamal. This variation is offers also additive homomorphism.

% \subsubsection{Twisted El Gamal}
\label{TwistedElGamal}

The differences between ElGamal and twisted ElGamal encryption is that in the latter: (i) the message is encoded over another generator $h$ that the logarithmic relation with $g$ is unknown (ii) the key encapsulation and session key are switched. It is been proved in \cite{PGC} that is as secure and efficient as standard ElGamal as well as it's ciphertext can be used as direct input to bulletproofs.

More specifically twisted ElGamal consists of the following algorithms:
\begin{itemize}
    \item $\setup(1^\lambda)$: run $(\group, g, p) \gets GroupGen(1^\lambda)$, pick $h \sample \group^*$, set $pp = (\group, g,h,p)$.
    \item $\kgen(pp)$: choose $\sk \sample \ZZ_p$, set $\pk = g^\sk$. 
    \item $\enc(\pk,\msg;r)$: compute $X = \pk^r, Y=g^r h^m$, output $C=(X,Y)$.
    \item $\dec(\sk, C)$: parse $C=(X,Y)$, compute $h^m = Y/X^{sk^{-1}}$. Recover $m$ from $h^m$. 
\end{itemize}
